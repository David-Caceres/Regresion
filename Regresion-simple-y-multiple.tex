% Options for packages loaded elsewhere
\PassOptionsToPackage{unicode}{hyperref}
\PassOptionsToPackage{hyphens}{url}
%
\documentclass[
]{article}
\title{Ejercicios\_Regresión\_simple\_multiple}
\author{David Cáceres}
\date{11/4/2022}

\usepackage{amsmath,amssymb}
\usepackage{lmodern}
\usepackage{iftex}
\ifPDFTeX
  \usepackage[T1]{fontenc}
  \usepackage[utf8]{inputenc}
  \usepackage{textcomp} % provide euro and other symbols
\else % if luatex or xetex
  \usepackage{unicode-math}
  \defaultfontfeatures{Scale=MatchLowercase}
  \defaultfontfeatures[\rmfamily]{Ligatures=TeX,Scale=1}
\fi
% Use upquote if available, for straight quotes in verbatim environments
\IfFileExists{upquote.sty}{\usepackage{upquote}}{}
\IfFileExists{microtype.sty}{% use microtype if available
  \usepackage[]{microtype}
  \UseMicrotypeSet[protrusion]{basicmath} % disable protrusion for tt fonts
}{}
\makeatletter
\@ifundefined{KOMAClassName}{% if non-KOMA class
  \IfFileExists{parskip.sty}{%
    \usepackage{parskip}
  }{% else
    \setlength{\parindent}{0pt}
    \setlength{\parskip}{6pt plus 2pt minus 1pt}}
}{% if KOMA class
  \KOMAoptions{parskip=half}}
\makeatother
\usepackage{xcolor}
\IfFileExists{xurl.sty}{\usepackage{xurl}}{} % add URL line breaks if available
\IfFileExists{bookmark.sty}{\usepackage{bookmark}}{\usepackage{hyperref}}
\hypersetup{
  pdftitle={Ejercicios\_Regresión\_simple\_multiple},
  pdfauthor={David Cáceres},
  hidelinks,
  pdfcreator={LaTeX via pandoc}}
\urlstyle{same} % disable monospaced font for URLs
\usepackage[margin=1in]{geometry}
\usepackage{color}
\usepackage{fancyvrb}
\newcommand{\VerbBar}{|}
\newcommand{\VERB}{\Verb[commandchars=\\\{\}]}
\DefineVerbatimEnvironment{Highlighting}{Verbatim}{commandchars=\\\{\}}
% Add ',fontsize=\small' for more characters per line
\usepackage{framed}
\definecolor{shadecolor}{RGB}{248,248,248}
\newenvironment{Shaded}{\begin{snugshade}}{\end{snugshade}}
\newcommand{\AlertTok}[1]{\textcolor[rgb]{0.94,0.16,0.16}{#1}}
\newcommand{\AnnotationTok}[1]{\textcolor[rgb]{0.56,0.35,0.01}{\textbf{\textit{#1}}}}
\newcommand{\AttributeTok}[1]{\textcolor[rgb]{0.77,0.63,0.00}{#1}}
\newcommand{\BaseNTok}[1]{\textcolor[rgb]{0.00,0.00,0.81}{#1}}
\newcommand{\BuiltInTok}[1]{#1}
\newcommand{\CharTok}[1]{\textcolor[rgb]{0.31,0.60,0.02}{#1}}
\newcommand{\CommentTok}[1]{\textcolor[rgb]{0.56,0.35,0.01}{\textit{#1}}}
\newcommand{\CommentVarTok}[1]{\textcolor[rgb]{0.56,0.35,0.01}{\textbf{\textit{#1}}}}
\newcommand{\ConstantTok}[1]{\textcolor[rgb]{0.00,0.00,0.00}{#1}}
\newcommand{\ControlFlowTok}[1]{\textcolor[rgb]{0.13,0.29,0.53}{\textbf{#1}}}
\newcommand{\DataTypeTok}[1]{\textcolor[rgb]{0.13,0.29,0.53}{#1}}
\newcommand{\DecValTok}[1]{\textcolor[rgb]{0.00,0.00,0.81}{#1}}
\newcommand{\DocumentationTok}[1]{\textcolor[rgb]{0.56,0.35,0.01}{\textbf{\textit{#1}}}}
\newcommand{\ErrorTok}[1]{\textcolor[rgb]{0.64,0.00,0.00}{\textbf{#1}}}
\newcommand{\ExtensionTok}[1]{#1}
\newcommand{\FloatTok}[1]{\textcolor[rgb]{0.00,0.00,0.81}{#1}}
\newcommand{\FunctionTok}[1]{\textcolor[rgb]{0.00,0.00,0.00}{#1}}
\newcommand{\ImportTok}[1]{#1}
\newcommand{\InformationTok}[1]{\textcolor[rgb]{0.56,0.35,0.01}{\textbf{\textit{#1}}}}
\newcommand{\KeywordTok}[1]{\textcolor[rgb]{0.13,0.29,0.53}{\textbf{#1}}}
\newcommand{\NormalTok}[1]{#1}
\newcommand{\OperatorTok}[1]{\textcolor[rgb]{0.81,0.36,0.00}{\textbf{#1}}}
\newcommand{\OtherTok}[1]{\textcolor[rgb]{0.56,0.35,0.01}{#1}}
\newcommand{\PreprocessorTok}[1]{\textcolor[rgb]{0.56,0.35,0.01}{\textit{#1}}}
\newcommand{\RegionMarkerTok}[1]{#1}
\newcommand{\SpecialCharTok}[1]{\textcolor[rgb]{0.00,0.00,0.00}{#1}}
\newcommand{\SpecialStringTok}[1]{\textcolor[rgb]{0.31,0.60,0.02}{#1}}
\newcommand{\StringTok}[1]{\textcolor[rgb]{0.31,0.60,0.02}{#1}}
\newcommand{\VariableTok}[1]{\textcolor[rgb]{0.00,0.00,0.00}{#1}}
\newcommand{\VerbatimStringTok}[1]{\textcolor[rgb]{0.31,0.60,0.02}{#1}}
\newcommand{\WarningTok}[1]{\textcolor[rgb]{0.56,0.35,0.01}{\textbf{\textit{#1}}}}
\usepackage{graphicx}
\makeatletter
\def\maxwidth{\ifdim\Gin@nat@width>\linewidth\linewidth\else\Gin@nat@width\fi}
\def\maxheight{\ifdim\Gin@nat@height>\textheight\textheight\else\Gin@nat@height\fi}
\makeatother
% Scale images if necessary, so that they will not overflow the page
% margins by default, and it is still possible to overwrite the defaults
% using explicit options in \includegraphics[width, height, ...]{}
\setkeys{Gin}{width=\maxwidth,height=\maxheight,keepaspectratio}
% Set default figure placement to htbp
\makeatletter
\def\fps@figure{htbp}
\makeatother
\setlength{\emergencystretch}{3em} % prevent overfull lines
\providecommand{\tightlist}{%
  \setlength{\itemsep}{0pt}\setlength{\parskip}{0pt}}
\setcounter{secnumdepth}{-\maxdimen} % remove section numbering
\ifLuaTeX
  \usepackage{selnolig}  % disable illegal ligatures
\fi

\begin{document}
\maketitle

\hypertarget{ejercicio-6.8-del-capuxedtulo-6-puxe1gina-118}{%
\paragraph{1. (Ejercicio 6.8 del Capítulo 6 página
118)}\label{ejercicio-6.8-del-capuxedtulo-6-puxe1gina-118}}

\hypertarget{hallar-la-recta-de-regresiuxf3n-simple-de-la-variable-respuesta-rauxedz-cuadrada-de-la-velocidad-sobre-la-variable-regresora-densidad-con-los-datos-de-la-tabla-1.1-del-capuxedtulo-1.}{%
\paragraph{Hallar la recta de regresión simple de la variable respuesta
raíz cuadrada de la velocidad sobre la variable regresora densidad con
los datos de la tabla 1.1 del capítulo
1.}\label{hallar-la-recta-de-regresiuxf3n-simple-de-la-variable-respuesta-rauxedz-cuadrada-de-la-velocidad-sobre-la-variable-regresora-densidad-con-los-datos-de-la-tabla-1.1-del-capuxedtulo-1.}}

\begin{Shaded}
\begin{Highlighting}[]
\FunctionTok{require}\NormalTok{(faraway)}
\end{Highlighting}
\end{Shaded}

\begin{verbatim}
## Loading required package: faraway
\end{verbatim}

\begin{Shaded}
\begin{Highlighting}[]
\NormalTok{dens }\OtherTok{\textless{}{-}} \FunctionTok{c}\NormalTok{(}\FloatTok{12.7}\NormalTok{,}\FloatTok{17.0}\NormalTok{,}\FloatTok{66.0}\NormalTok{,}\FloatTok{50.0}\NormalTok{,}\FloatTok{87.8}\NormalTok{,}\FloatTok{81.4}\NormalTok{,}\FloatTok{75.6}\NormalTok{,}\FloatTok{66.2}\NormalTok{,}\FloatTok{81.1}\NormalTok{,}\FloatTok{62.8}\NormalTok{,}\FloatTok{77.0}\NormalTok{,}\FloatTok{89.6}\NormalTok{,}
           \FloatTok{18.3}\NormalTok{,}\FloatTok{19.1}\NormalTok{,}\FloatTok{16.5}\NormalTok{,}\FloatTok{22.2}\NormalTok{,}\FloatTok{18.6}\NormalTok{,}\FloatTok{66.0}\NormalTok{,}\FloatTok{60.3}\NormalTok{,}\FloatTok{56.0}\NormalTok{,}\FloatTok{66.3}\NormalTok{,}\FloatTok{61.7}\NormalTok{,}\FloatTok{66.6}\NormalTok{,}\FloatTok{67.8}\NormalTok{)}
\NormalTok{vel }\OtherTok{\textless{}{-}} \FunctionTok{c}\NormalTok{(}\FloatTok{62.4}\NormalTok{,}\FloatTok{50.7}\NormalTok{,}\FloatTok{17.1}\NormalTok{,}\FloatTok{25.9}\NormalTok{,}\FloatTok{12.4}\NormalTok{,}\FloatTok{13.4}\NormalTok{,}\FloatTok{13.7}\NormalTok{,}\FloatTok{17.9}\NormalTok{,}\FloatTok{13.8}\NormalTok{,}\FloatTok{17.9}\NormalTok{,}\FloatTok{15.8}\NormalTok{,}\FloatTok{12.6}\NormalTok{,}
          \FloatTok{51.2}\NormalTok{,}\FloatTok{50.8}\NormalTok{,}\FloatTok{54.7}\NormalTok{,}\FloatTok{46.5}\NormalTok{,}\FloatTok{46.3}\NormalTok{,}\FloatTok{16.9}\NormalTok{,}\FloatTok{19.8}\NormalTok{,}\FloatTok{21.2}\NormalTok{,}\FloatTok{18.3}\NormalTok{,}\FloatTok{18.0}\NormalTok{,}\FloatTok{16.6}\NormalTok{,}\FloatTok{18.3}\NormalTok{)}
\NormalTok{rvel }\OtherTok{\textless{}{-}} \FunctionTok{sqrt}\NormalTok{(vel)}
\NormalTok{lmod1}\OtherTok{\textless{}{-}}\FunctionTok{lm}\NormalTok{(rvel }\SpecialCharTok{\textasciitilde{}}\NormalTok{ dens)}
\FunctionTok{summary}\NormalTok{(lmod1)}
\end{Highlighting}
\end{Shaded}

\begin{verbatim}
## 
## Call:
## lm(formula = rvel ~ dens)
## 
## Residuals:
##      Min       1Q   Median       3Q      Max 
## -0.35337 -0.22722 -0.03566  0.18942  0.53349 
## 
## Coefficients:
##              Estimate Std. Error t value Pr(>|t|)    
## (Intercept)  8.089813   0.130629   61.93   <2e-16 ***
## dens        -0.056626   0.002177  -26.01   <2e-16 ***
## ---
## Signif. codes:  0 '***' 0.001 '**' 0.01 '*' 0.05 '.' 0.1 ' ' 1
## 
## Residual standard error: 0.2689 on 22 degrees of freedom
## Multiple R-squared:  0.9685, Adjusted R-squared:  0.9671 
## F-statistic: 676.4 on 1 and 22 DF,  p-value: < 2.2e-16
\end{verbatim}

\hypertarget{i-la-suma-de-los-residuos-es-cero}{%
\paragraph{(i) La suma de los residuos es
cero}\label{i-la-suma-de-los-residuos-es-cero}}

\begin{Shaded}
\begin{Highlighting}[]
\NormalTok{residuos}\OtherTok{\textless{}{-}} \FunctionTok{residuals}\NormalTok{(lmod1)}
\FunctionTok{sum}\NormalTok{(residuos)}
\end{Highlighting}
\end{Shaded}

\begin{verbatim}
## [1] 8.881784e-16
\end{verbatim}

\begin{Shaded}
\begin{Highlighting}[]
\CommentTok{\# El valor es prácticamente cero por redondeo}
\end{Highlighting}
\end{Shaded}

\hypertarget{ii-pyi-pyux2c6i}{%
\paragraph{(ii) Pyi = Pyˆi}\label{ii-pyi-pyux2c6i}}

\begin{Shaded}
\begin{Highlighting}[]
\FunctionTok{sum}\NormalTok{(rvel)}
\end{Highlighting}
\end{Shaded}

\begin{verbatim}
## [1] 120.1685
\end{verbatim}

\begin{Shaded}
\begin{Highlighting}[]
\FunctionTok{sum}\NormalTok{(}\FunctionTok{fitted}\NormalTok{(lmod1))}
\end{Highlighting}
\end{Shaded}

\begin{verbatim}
## [1] 120.1685
\end{verbatim}

\hypertarget{iii-la-suma-de-los-residuos-ponderada-por-los-valores-de-la-variable-regresora-es-cero}{%
\paragraph{(iii) La suma de los residuos ponderada por los valores de la
variable regresora es
cero:}\label{iii-la-suma-de-los-residuos-ponderada-por-los-valores-de-la-variable-regresora-es-cero}}

\begin{Shaded}
\begin{Highlighting}[]
\FunctionTok{sum}\NormalTok{(residuos}\SpecialCharTok{*}\NormalTok{dens)}
\end{Highlighting}
\end{Shaded}

\begin{verbatim}
## [1] -1.566802e-14
\end{verbatim}

\hypertarget{iv-la-suma-de-los-residuos-ponderada-por-las-predicciones-de-los-valores-observados-es-cero}{%
\paragraph{(iv) La suma de los residuos ponderada por las predicciones
de los valores observados es
cero:}\label{iv-la-suma-de-los-residuos-ponderada-por-las-predicciones-de-los-valores-observados-es-cero}}

\begin{Shaded}
\begin{Highlighting}[]
\FunctionTok{sum}\NormalTok{(}\FunctionTok{fitted}\NormalTok{(lmod1)}\SpecialCharTok{*}\NormalTok{residuos)}
\end{Highlighting}
\end{Shaded}

\begin{verbatim}
## [1] 2.491063e-15
\end{verbatim}

\hypertarget{calcular-la-estimaciuxf3n-de-ux3c3uxb2-y-a-partir-de-ella-las-estimaciones-de-las-desviaciones-estuxe1ndar-de-los-estimadores-de-los-paruxe1metros-ux3b2ux2c60-y-ux3b2ux2c61.}{%
\paragraph{Calcular la estimación de σ² y, a partir de ella, las
estimaciones de las desviaciones estándar de los estimadores de los
parámetros βˆ0 y
βˆ1.}\label{calcular-la-estimaciuxf3n-de-ux3c3uxb2-y-a-partir-de-ella-las-estimaciones-de-las-desviaciones-estuxe1ndar-de-los-estimadores-de-los-paruxe1metros-ux3b2ux2c60-y-ux3b2ux2c61.}}

\begin{Shaded}
\begin{Highlighting}[]
\NormalTok{length}\OtherTok{\textless{}{-}}\FunctionTok{length}\NormalTok{(dens)}
\NormalTok{SCRlmod1}\OtherTok{\textless{}{-}}\FunctionTok{sum}\NormalTok{(residuos}\SpecialCharTok{\^{}}\DecValTok{2}\NormalTok{)}
\NormalTok{sigma}\OtherTok{\textless{}{-}}\NormalTok{SCRlmod1}\SpecialCharTok{/}\NormalTok{(length}\DecValTok{{-}2}\NormalTok{)}
\NormalTok{sigma}
\end{Highlighting}
\end{Shaded}

\begin{verbatim}
## [1] 0.07232808
\end{verbatim}

\begin{Shaded}
\begin{Highlighting}[]
\NormalTok{s2dens}\OtherTok{\textless{}{-}}\NormalTok{(}\DecValTok{1}\SpecialCharTok{/}\NormalTok{length)}\SpecialCharTok{*}\FunctionTok{sum}\NormalTok{((dens}\SpecialCharTok{{-}}\FunctionTok{mean}\NormalTok{(dens))}\SpecialCharTok{\^{}}\DecValTok{2}\NormalTok{)}
\NormalTok{Sdens}\OtherTok{\textless{}{-}}\NormalTok{length}\SpecialCharTok{*}\NormalTok{s2dens}
\NormalTok{ebeta1lmod1}\OtherTok{\textless{}{-}}\FunctionTok{sqrt}\NormalTok{(sigma}\SpecialCharTok{*}\NormalTok{((}\DecValTok{1}\SpecialCharTok{/}\NormalTok{length)}\SpecialCharTok{+}\NormalTok{(}\FunctionTok{mean}\NormalTok{(dens)}\SpecialCharTok{\^{}}\DecValTok{2}\SpecialCharTok{/}\NormalTok{Sdens)))}
\end{Highlighting}
\end{Shaded}

\begin{Shaded}
\begin{Highlighting}[]
\NormalTok{ebeta0lmod1}\OtherTok{\textless{}{-}}\FunctionTok{sqrt}\NormalTok{(sigma}\SpecialCharTok{*}\NormalTok{((}\DecValTok{1}\SpecialCharTok{/}\NormalTok{length)}\SpecialCharTok{+}\NormalTok{(}\FunctionTok{mean}\NormalTok{(dens)}\SpecialCharTok{\^{}}\DecValTok{2}\SpecialCharTok{/}\NormalTok{Sdens)))}
\NormalTok{ebeta0lmod1}
\end{Highlighting}
\end{Shaded}

\begin{verbatim}
## [1] 0.1306294
\end{verbatim}

\hypertarget{escribir-los-intervalos-de-confianza-para-los-paruxe1metros-con-un-nivel-de-confianza-del-95-.}{%
\paragraph{Escribir los intervalos de confianza para los parámetros con
un nivel de confianza del 95
\%.}\label{escribir-los-intervalos-de-confianza-para-los-paruxe1metros-con-un-nivel-de-confianza-del-95-.}}

\begin{Shaded}
\begin{Highlighting}[]
\NormalTok{tlmod1 }\OtherTok{\textless{}{-}} \FunctionTok{qt}\NormalTok{(}\FloatTok{0.975}\NormalTok{,lmod1}\SpecialCharTok{$}\NormalTok{df)}
\CommentTok{\# Para Beta1}
\NormalTok{lmod1}\SpecialCharTok{$}\NormalTok{coef[}\DecValTok{2}\NormalTok{]}\SpecialCharTok{+}\FunctionTok{c}\NormalTok{(}\SpecialCharTok{{-}}\DecValTok{1}\NormalTok{,}\DecValTok{1}\NormalTok{)}\SpecialCharTok{*}\NormalTok{tlmod1}\SpecialCharTok{*}\NormalTok{ebeta1lmod1}
\end{Highlighting}
\end{Shaded}

\begin{verbatim}
## [1] -0.3275345  0.2142833
\end{verbatim}

\begin{Shaded}
\begin{Highlighting}[]
\CommentTok{\# Para beta0}
\NormalTok{lmod1}\SpecialCharTok{$}\NormalTok{coef[}\DecValTok{1}\NormalTok{]}\SpecialCharTok{+}\FunctionTok{c}\NormalTok{(}\SpecialCharTok{{-}}\DecValTok{1}\NormalTok{,}\DecValTok{1}\NormalTok{)}\SpecialCharTok{*}\NormalTok{tlmod1}\SpecialCharTok{*}\NormalTok{ebeta0lmod1}
\end{Highlighting}
\end{Shaded}

\begin{verbatim}
## [1] 7.818904 8.360722
\end{verbatim}

\hypertarget{construir-la-tabla-para-la-significaciuxf3n-de-la-regresiuxf3n-y-realizar-dicho-contraste.-h0-uxdf10-no-hay-efecto-de-la-relaciuxf3n-entre-ambas-variables}{%
\paragraph{Construir la tabla para la significación de la regresión y
realizar dicho contraste. (H0: ß1=0 (No hay efecto de la relación entre
ambas
variables)}\label{construir-la-tabla-para-la-significaciuxf3n-de-la-regresiuxf3n-y-realizar-dicho-contraste.-h0-uxdf10-no-hay-efecto-de-la-relaciuxf3n-entre-ambas-variables}}

\begin{Shaded}
\begin{Highlighting}[]
\FunctionTok{summary}\NormalTok{(}\FunctionTok{aov}\NormalTok{(lmod1))}
\end{Highlighting}
\end{Shaded}

\begin{verbatim}
##             Df Sum Sq Mean Sq F value Pr(>F)    
## dens         1  48.92   48.92   676.4 <2e-16 ***
## Residuals   22   1.59    0.07                   
## ---
## Signif. codes:  0 '***' 0.001 '**' 0.01 '*' 0.05 '.' 0.1 ' ' 1
\end{verbatim}

Significativo a p\textless\textless alfa=0.005

\hypertarget{hallar-el-intervalo-de-la-predicciuxf3n-de-la-respuesta-media-cuando-la-densidades-de-50-vehuxedculos-por-km.-nivel-de-confianza-90}\label{hallar-el-intervalo-de-la-predicciuxf3n-de-la-respuesta-media-cuando-la-densidades-de-50-vehuxedculos-por-km.-nivel-de-confianza-90}}

\begin{Shaded}
\begin{Highlighting}[]
\NormalTok{y0 }\OtherTok{\textless{}{-}}\NormalTok{ lmod1}\SpecialCharTok{$}\NormalTok{coef[}\DecValTok{1}\NormalTok{]}\SpecialCharTok{+}\NormalTok{lmod1}\SpecialCharTok{$}\NormalTok{coef[}\DecValTok{2}\NormalTok{]}\SpecialCharTok{*}\DecValTok{50}

\NormalTok{y0}
\end{Highlighting}
\end{Shaded}

\begin{verbatim}
## (Intercept) 
##    5.258534
\end{verbatim}

\begin{Shaded}
\begin{Highlighting}[]
\NormalTok{tlmod1 }\OtherTok{\textless{}{-}} \FunctionTok{qt}\NormalTok{(}\FloatTok{0.950}\NormalTok{,lmod1}\SpecialCharTok{$}\NormalTok{df)}
\NormalTok{sigma }\OtherTok{\textless{}{-}} \FunctionTok{sqrt}\NormalTok{(sigma)}
\NormalTok{y0}\SpecialCharTok{+}\FunctionTok{c}\NormalTok{(}\SpecialCharTok{{-}}\DecValTok{1}\NormalTok{,}\DecValTok{1}\NormalTok{)}\SpecialCharTok{*}\NormalTok{tlmod1}\SpecialCharTok{*}\NormalTok{sigma}\SpecialCharTok{*}\FunctionTok{sqrt}\NormalTok{(}\DecValTok{1}\SpecialCharTok{+}\NormalTok{(}\DecValTok{1}\SpecialCharTok{/}\NormalTok{length)}\SpecialCharTok{+}\NormalTok{((((}\DecValTok{50}\SpecialCharTok{{-}}\FunctionTok{mean}\NormalTok{(dens))}\SpecialCharTok{\^{}}\DecValTok{2}\NormalTok{)}\SpecialCharTok{/}\NormalTok{Sdens)))}
\end{Highlighting}
\end{Shaded}

\begin{verbatim}
## [1] 4.786912 5.730156
\end{verbatim}

\begin{Shaded}
\begin{Highlighting}[]
\FunctionTok{predict}\NormalTok{(lmod1,}\AttributeTok{new=}\FunctionTok{data.frame}\NormalTok{(}\AttributeTok{dens=}\DecValTok{50}\NormalTok{),}\AttributeTok{interval=}\StringTok{"prediction"}\NormalTok{,}\AttributeTok{level=}\FloatTok{0.90}\NormalTok{)}
\end{Highlighting}
\end{Shaded}

\begin{verbatim}
##        fit      lwr      upr
## 1 5.258534 4.786912 5.730156
\end{verbatim}

\begin{Shaded}
\begin{Highlighting}[]
 \FunctionTok{predict}\NormalTok{(lmod1,}\AttributeTok{new=}\FunctionTok{data.frame}\NormalTok{(}\AttributeTok{dens=}\DecValTok{50}\NormalTok{),}\AttributeTok{interval=}\StringTok{"confidence"}\NormalTok{,}\AttributeTok{level=}\FloatTok{0.90}\NormalTok{)}
\end{Highlighting}
\end{Shaded}

\begin{verbatim}
##        fit      lwr      upr
## 1 5.258534 5.162817 5.354251
\end{verbatim}

\hypertarget{ejercicio-6.10-del-capuxedtulo-6-puxe1gina-118}{%
\subsubsection{3. (Ejercicio 6.10 del Capítulo 6 página
118)}\label{ejercicio-6.10-del-capuxedtulo-6-puxe1gina-118}}

\#\#\#\#Se admite que una persona es proporcionada si su altura en cm es
igual a su peso en kg más 100. En términos estadísticos si la recta de
regresión de Y (altura) sobre X (peso) es:

Y = 100 + X

\hypertarget{contrastar-con-un-nivel-de-significaciuxf3n-ux3b1-0.05-si-se-puede-considerar-vuxe1lida-esta-hipuxf3tesis-a-partir-de-los-siguientes-datos-que-corresponden-a-una-muestra-de-mujeres-juxf3venes}{%
\paragraph{Contrastar, con un nivel de significación α = 0.05, si se
puede considerar válida esta hipótesis a partir de los siguientes datos
que corresponden a una muestra de mujeres
jóvenes:}\label{contrastar-con-un-nivel-de-significaciuxf3n-ux3b1-0.05-si-se-puede-considerar-vuxe1lida-esta-hipuxf3tesis-a-partir-de-los-siguientes-datos-que-corresponden-a-una-muestra-de-mujeres-juxf3venes}}

\begin{Shaded}
\begin{Highlighting}[]
\NormalTok{x }\OtherTok{\textless{}{-}} \FunctionTok{c}\NormalTok{(}\DecValTok{55}\NormalTok{,}\DecValTok{52}\NormalTok{,}\DecValTok{65}\NormalTok{,}\DecValTok{54}\NormalTok{,}\DecValTok{46}\NormalTok{,}\DecValTok{60}\NormalTok{,}\DecValTok{54}\NormalTok{,}\DecValTok{52}\NormalTok{,}\DecValTok{56}\NormalTok{,}\DecValTok{65}\NormalTok{,}\DecValTok{52}\NormalTok{,}\DecValTok{53}\NormalTok{,}\DecValTok{60}\NormalTok{)}
\NormalTok{y }\OtherTok{\textless{}{-}} \FunctionTok{c}\NormalTok{(}\DecValTok{164}\NormalTok{,}\DecValTok{164}\NormalTok{,}\DecValTok{173}\NormalTok{,}\DecValTok{163}\NormalTok{,}\DecValTok{157}\NormalTok{,}\DecValTok{168}\NormalTok{,}\DecValTok{171}\NormalTok{,}\DecValTok{158}\NormalTok{,}\DecValTok{169}\NormalTok{,}\DecValTok{172}\NormalTok{,}\DecValTok{168}\NormalTok{,}\DecValTok{160}\NormalTok{,}\DecValTok{172}\NormalTok{)}
\end{Highlighting}
\end{Shaded}

\hypertarget{razonar-la-bondad-de-la-regresiuxf3n-y-todos-los-detalles-del-contraste.}{%
\paragraph{Razonar la bondad de la regresión y todos los detalles del
contraste.}\label{razonar-la-bondad-de-la-regresiuxf3n-y-todos-los-detalles-del-contraste.}}

\begin{Shaded}
\begin{Highlighting}[]
\NormalTok{lmod2}\OtherTok{\textless{}{-}}\FunctionTok{lm}\NormalTok{(y  }\SpecialCharTok{\textasciitilde{}}\NormalTok{ x)}
\FunctionTok{summary}\NormalTok{(lmod2)}
\end{Highlighting}
\end{Shaded}

\begin{verbatim}
## 
## Call:
## lm(formula = y ~ x)
## 
## Residuals:
##    Min     1Q Median     3Q    Max 
## -5.120 -1.531 -1.315  2.473  6.278 
## 
## Coefficients:
##             Estimate Std. Error t value Pr(>|t|)    
## (Intercept) 121.4777    10.1857  11.926 1.24e-07 ***
## x             0.8008     0.1821   4.398  0.00107 ** 
## ---
## Signif. codes:  0 '***' 0.001 '**' 0.01 '*' 0.05 '.' 0.1 ' ' 1
## 
## Residual standard error: 3.449 on 11 degrees of freedom
## Multiple R-squared:  0.6375, Adjusted R-squared:  0.6045 
## F-statistic: 19.34 on 1 and 11 DF,  p-value: 0.001067
\end{verbatim}

\begin{Shaded}
\begin{Highlighting}[]
\FunctionTok{plot}\NormalTok{(y}\SpecialCharTok{\textasciitilde{}}\NormalTok{x, }\AttributeTok{xlim=}\FunctionTok{c}\NormalTok{(}\DecValTok{40}\NormalTok{,}\DecValTok{75}\NormalTok{), }\AttributeTok{ylim=}\FunctionTok{c}\NormalTok{(}\DecValTok{140}\NormalTok{, }\DecValTok{180}\NormalTok{))}
\FunctionTok{abline}\NormalTok{(lmod2)}
\FunctionTok{abline}\NormalTok{(}\DecValTok{100}\NormalTok{,}\DecValTok{1}\NormalTok{)}
\end{Highlighting}
\end{Shaded}

\includegraphics{Regresion-simple-y-multiple_files/figure-latex/unnamed-chunk-14-1.pdf}

\begin{Shaded}
\begin{Highlighting}[]
\FunctionTok{confint}\NormalTok{(lmod2)}
\end{Highlighting}
\end{Shaded}

\begin{verbatim}
##                  2.5 %     97.5 %
## (Intercept) 99.0591808 143.896222
## x            0.4000513   1.201578
\end{verbatim}

\begin{Shaded}
\begin{Highlighting}[]
\NormalTok{modelo0}\OtherTok{\textless{}{-}}\FunctionTok{lm}\NormalTok{(y }\SpecialCharTok{\textasciitilde{}} \DecValTok{0} \SpecialCharTok{+} \FunctionTok{offset}\NormalTok{(}\DecValTok{100} \SpecialCharTok{+}\NormalTok{ x))}
\FunctionTok{anova}\NormalTok{(modelo0, lmod2)}
\end{Highlighting}
\end{Shaded}

\begin{verbatim}
## Analysis of Variance Table
## 
## Model 1: y ~ 0 + offset(100 + x)
## Model 2: y ~ x
##   Res.Df     RSS Df Sum of Sq      F    Pr(>F)    
## 1     13 1547.00                                  
## 2     11  130.84  2    1416.2 59.528 1.259e-06 ***
## ---
## Signif. codes:  0 '***' 0.001 '**' 0.01 '*' 0.05 '.' 0.1 ' ' 1
\end{verbatim}

Rechazamos la hipótesis nula

\hypertarget{ejercicio-6.11-del-capuxedtulo-6-puxe1gina-119}{%
\paragraph{4. (Ejercicio 6.11 del Capítulo 6 página
119)}\label{ejercicio-6.11-del-capuxedtulo-6-puxe1gina-119}}

El período de oscilación de un péndulo es 2π\textbar{} l/g donde l es la
longitud y g es la constante de gravitación. En un experimento
observamos t ij (j = 1, . . . , n i ) períodos correspondientes a l i (i
= 1, . . . , k) longitudes.

\hypertarget{a-proponer-un-modelo-con-las-hipuxf3tesis-que-se-necesiten-para-estimar-la-constante-muxe9todo-de-los-muxednimos-cuadrados.}{%
\paragraph{(a) Proponer un modelo, con las hipótesis que se necesiten,
para estimar la constante método de los mínimos
cuadrados.}\label{a-proponer-un-modelo-con-las-hipuxf3tesis-que-se-necesiten-para-estimar-la-constante-muxe9todo-de-los-muxednimos-cuadrados.}}

\begin{Shaded}
\begin{Highlighting}[]
\NormalTok{longitud}\OtherTok{\textless{}{-}} \FunctionTok{c}\NormalTok{(}\FunctionTok{rep}\NormalTok{(}\FloatTok{18.3}\NormalTok{,}\DecValTok{4}\NormalTok{),}\FunctionTok{rep}\NormalTok{(}\DecValTok{20}\NormalTok{,}\DecValTok{2}\NormalTok{),}\FunctionTok{rep}\NormalTok{(}\FloatTok{21.5}\NormalTok{,}\DecValTok{3}\NormalTok{),}\FunctionTok{rep}\NormalTok{(}\DecValTok{15}\NormalTok{,}\DecValTok{2}\NormalTok{))}
\NormalTok{x}\OtherTok{\textless{}{-}}\FunctionTok{sqrt}\NormalTok{(longitud)}
\NormalTok{y}\OtherTok{\textless{}{-}}\FunctionTok{c}\NormalTok{(}\FloatTok{8.58}\NormalTok{,}\FloatTok{7.9}\NormalTok{,}\FloatTok{8.2}\NormalTok{,}\FloatTok{7.8}\NormalTok{,}\FloatTok{8.4}\NormalTok{,}\FloatTok{9.2}\NormalTok{,}\FloatTok{9.7}\NormalTok{,}\FloatTok{8.95}\NormalTok{,}\FloatTok{9.2}\NormalTok{,}\FloatTok{7.5}\NormalTok{,}\DecValTok{8}\NormalTok{)}
\NormalTok{lmod3}\OtherTok{\textless{}{-}}\FunctionTok{lm}\NormalTok{(y }\SpecialCharTok{\textasciitilde{}} \DecValTok{0} \SpecialCharTok{+} \FunctionTok{sqrt}\NormalTok{(x))}
\FunctionTok{summary}\NormalTok{(lmod3)}
\end{Highlighting}
\end{Shaded}

\begin{verbatim}
## 
## Call:
## lm(formula = y ~ 0 + sqrt(x))
## 
## Residuals:
##      Min       1Q   Median       3Q      Max 
## -0.64760 -0.39276 -0.03791  0.28012  0.90512 
## 
## Coefficients:
##         Estimate Std. Error t value Pr(>|t|)    
## sqrt(x)  4.08433    0.07194   56.77 6.97e-14 ***
## ---
## Signif. codes:  0 '***' 0.001 '**' 0.01 '*' 0.05 '.' 0.1 ' ' 1
## 
## Residual standard error: 0.4969 on 10 degrees of freedom
## Multiple R-squared:  0.9969, Adjusted R-squared:  0.9966 
## F-statistic:  3223 on 1 and 10 DF,  p-value: 6.973e-14
\end{verbatim}

\hypertarget{b-contrastar-h02ux3c0g}{%
\paragraph{(b) Contrastar H0:2πg√}\label{b-contrastar-h02ux3c0g}}

\begin{Shaded}
\begin{Highlighting}[]
\CommentTok{\# Miramos si el intervalo de confianza contiene el 2}

\FunctionTok{confint}\NormalTok{(lmod3)}
\end{Highlighting}
\end{Shaded}

\begin{verbatim}
##            2.5 %   97.5 %
## sqrt(x) 3.924036 4.244618
\end{verbatim}

No lo contiene así que rechazamos la hipótesis nula

\hypertarget{ejercicio-8.4-del-capuxedtulo-8-puxe1gina-157}{%
\paragraph{5. (Ejercicio 8.4 del Capítulo 8 página
157)}\label{ejercicio-8.4-del-capuxedtulo-8-puxe1gina-157}}

Se dispone de los siguientes datos sobre diez empresas fabricantes de
productos de limpieza doméstica:

\begin{Shaded}
\begin{Highlighting}[]
\CommentTok{\# Datos}

\NormalTok{v }\OtherTok{\textless{}{-}} \FunctionTok{c}\NormalTok{(}\DecValTok{60}\NormalTok{,}\DecValTok{48}\NormalTok{,}\DecValTok{42}\NormalTok{,}\DecValTok{36}\NormalTok{,}\DecValTok{78}\NormalTok{,}\DecValTok{36}\NormalTok{,}\DecValTok{72}\NormalTok{,}\DecValTok{42}\NormalTok{,}\DecValTok{54}\NormalTok{,}\DecValTok{90}\NormalTok{)}
\NormalTok{ip }\OtherTok{\textless{}{-}} \FunctionTok{c}\NormalTok{(}\DecValTok{100}\NormalTok{,}\DecValTok{110}\NormalTok{,}\DecValTok{130}\NormalTok{,}\DecValTok{100}\NormalTok{,}\DecValTok{80}\NormalTok{,}\DecValTok{80}\NormalTok{,}\DecValTok{90}\NormalTok{,}\DecValTok{120}\NormalTok{,}\DecValTok{120}\NormalTok{,}\DecValTok{90}\NormalTok{)}
\NormalTok{pu }\OtherTok{\textless{}{-}} \FunctionTok{c}\NormalTok{(}\FloatTok{1.8}\NormalTok{,}\FloatTok{2.4}\NormalTok{,}\FloatTok{3.6}\NormalTok{,}\FloatTok{0.6}\NormalTok{,}\FloatTok{1.8}\NormalTok{,}\FloatTok{0.6}\NormalTok{,}\FloatTok{3.6}\NormalTok{,}\FloatTok{1.2}\NormalTok{,}\FloatTok{2.4}\NormalTok{,}\FloatTok{4.2}\NormalTok{)}
\end{Highlighting}
\end{Shaded}

\begin{Shaded}
\begin{Highlighting}[]
\NormalTok{lmod4}\OtherTok{\textless{}{-}}\FunctionTok{lm}\NormalTok{(v }\SpecialCharTok{\textasciitilde{}}\NormalTok{ ip }\SpecialCharTok{+}\NormalTok{pu)}
\FunctionTok{summary}\NormalTok{(lmod4)}
\end{Highlighting}
\end{Shaded}

\begin{verbatim}
## 
## Call:
## lm(formula = v ~ ip + pu)
## 
## Residuals:
##     Min      1Q  Median      3Q     Max 
## -15.865  -5.944   0.869   7.516  13.051 
## 
## Coefficients:
##             Estimate Std. Error t value Pr(>|t|)   
## (Intercept)  95.2462    21.7077   4.388  0.00320 **
## ip           -0.6240     0.2112  -2.954  0.02128 * 
## pu           10.9038     2.9208   3.733  0.00733 **
## ---
## Signif. codes:  0 '***' 0.001 '**' 0.01 '*' 0.05 '.' 0.1 ' ' 1
## 
## Residual standard error: 10.94 on 7 degrees of freedom
## Multiple R-squared:  0.7359, Adjusted R-squared:  0.6604 
## F-statistic: 9.752 on 2 and 7 DF,  p-value: 0.009469
\end{verbatim}

\hypertarget{estimar-el-vector-de-coeficientes-ux3b2-ux3b2-0-ux3b2-1-ux3b2-2-0-del-modelo}{%
\paragraph{1) Estimar el vector de coeficientes β = (β 0 , β 1 , β 2 ) 0
del
modelo}\label{estimar-el-vector-de-coeficientes-ux3b2-ux3b2-0-ux3b2-1-ux3b2-2-0-del-modelo}}

\begin{Shaded}
\begin{Highlighting}[]
\NormalTok{lmod4}\SpecialCharTok{$}\NormalTok{coefficients}
\end{Highlighting}
\end{Shaded}

\begin{verbatim}
## (Intercept)          ip          pu 
##  95.2462340  -0.6240469  10.9038497
\end{verbatim}

\hypertarget{estimar-la-matriz-de-varianzas-covarianzas-del-vector-ux3b2.}{%
\paragraph{2) Estimar la matriz de varianzas-covarianzas del vector
β.}\label{estimar-la-matriz-de-varianzas-covarianzas-del-vector-ux3b2.}}

\begin{Shaded}
\begin{Highlighting}[]
\FunctionTok{summary}\NormalTok{(lmod4)}\SpecialCharTok{$}\NormalTok{sigma}\SpecialCharTok{\^{}}\DecValTok{2} \SpecialCharTok{*} \FunctionTok{summary}\NormalTok{(lmod4)}\SpecialCharTok{$}\NormalTok{cov.unscaled}
\end{Highlighting}
\end{Shaded}

\begin{verbatim}
##             (Intercept)          ip         pu
## (Intercept)  471.224820 -4.32088690 -8.3457769
## ip            -4.320887  0.04462209 -0.1038586
## pu            -8.345777 -0.10385856  8.5312386
\end{verbatim}

\hypertarget{calcular-el-coeficiente-de-determinaciuxf3n.}{%
\paragraph{3) Calcular el coeficiente de
determinación.}\label{calcular-el-coeficiente-de-determinaciuxf3n.}}

\begin{Shaded}
\begin{Highlighting}[]
\NormalTok{sum4}\OtherTok{\textless{}{-}}\FunctionTok{summary}\NormalTok{(lmod4)}
\FunctionTok{names}\NormalTok{(sum4)}
\end{Highlighting}
\end{Shaded}

\begin{verbatim}
##  [1] "call"          "terms"         "residuals"     "coefficients" 
##  [5] "aliased"       "sigma"         "df"            "r.squared"    
##  [9] "adj.r.squared" "fstatistic"    "cov.unscaled"
\end{verbatim}

\begin{Shaded}
\begin{Highlighting}[]
\NormalTok{sum4}\SpecialCharTok{$}\NormalTok{r.squared}
\end{Highlighting}
\end{Shaded}

\begin{verbatim}
## [1] 0.7358837
\end{verbatim}

\hypertarget{ejercicios-del-libro-de-faraway}{%
\subsection{Ejercicios del libro de
Faraway}\label{ejercicios-del-libro-de-faraway}}

\hypertarget{ejercicio-1-cap.-4-puxe1g.-56}{%
\paragraph{1. (Ejercicio 1 cap. 4 pág.
56)}\label{ejercicio-1-cap.-4-puxe1g.-56}}

For the prostate data, fit a model with lpsa as the response and the
other variables as predictors:

\begin{Shaded}
\begin{Highlighting}[]
\FunctionTok{require}\NormalTok{(faraway)}
\FunctionTok{data}\NormalTok{(prostate)}
\end{Highlighting}
\end{Shaded}

\begin{Shaded}
\begin{Highlighting}[]
\NormalTok{lmod5}\OtherTok{\textless{}{-}} \FunctionTok{lm}\NormalTok{(lpsa}\SpecialCharTok{\textasciitilde{}}\NormalTok{lcavol }\SpecialCharTok{+}\NormalTok{ lweight }\SpecialCharTok{+}\NormalTok{ age }\SpecialCharTok{+}\NormalTok{ lbph }\SpecialCharTok{+}\NormalTok{ svi }\SpecialCharTok{+}\NormalTok{lcp}\SpecialCharTok{+}\NormalTok{ gleason}\SpecialCharTok{+}\NormalTok{pgg45, }\AttributeTok{data =}\NormalTok{ prostate)}
\FunctionTok{summary}\NormalTok{(lmod5)}
\end{Highlighting}
\end{Shaded}

\begin{verbatim}
## 
## Call:
## lm(formula = lpsa ~ lcavol + lweight + age + lbph + svi + lcp + 
##     gleason + pgg45, data = prostate)
## 
## Residuals:
##     Min      1Q  Median      3Q     Max 
## -1.7331 -0.3713 -0.0170  0.4141  1.6381 
## 
## Coefficients:
##              Estimate Std. Error t value Pr(>|t|)    
## (Intercept)  0.669337   1.296387   0.516  0.60693    
## lcavol       0.587022   0.087920   6.677 2.11e-09 ***
## lweight      0.454467   0.170012   2.673  0.00896 ** 
## age         -0.019637   0.011173  -1.758  0.08229 .  
## lbph         0.107054   0.058449   1.832  0.07040 .  
## svi          0.766157   0.244309   3.136  0.00233 ** 
## lcp         -0.105474   0.091013  -1.159  0.24964    
## gleason      0.045142   0.157465   0.287  0.77503    
## pgg45        0.004525   0.004421   1.024  0.30886    
## ---
## Signif. codes:  0 '***' 0.001 '**' 0.01 '*' 0.05 '.' 0.1 ' ' 1
## 
## Residual standard error: 0.7084 on 88 degrees of freedom
## Multiple R-squared:  0.6548, Adjusted R-squared:  0.6234 
## F-statistic: 20.86 on 8 and 88 DF,  p-value: < 2.2e-16
\end{verbatim}

\hypertarget{a-suppose-a-new-patient-with-the-following-values-arrives}{%
\paragraph{(a) Suppose a new patient with the following values
arrives:}\label{a-suppose-a-new-patient-with-the-following-values-arrives}}

Predict the lpsa for this patient along with an appropriate 95\% CI.

\begin{Shaded}
\begin{Highlighting}[]
\FunctionTok{head}\NormalTok{(x0pros }\OtherTok{\textless{}{-}} \FunctionTok{data.frame}\NormalTok{(}\AttributeTok{lcavol=}\FloatTok{1.44692}\NormalTok{,}
                                              \AttributeTok{lweight=}\FloatTok{3.62301}\NormalTok{,}
                                              \AttributeTok{age=}\DecValTok{65}\NormalTok{,}
                                              \AttributeTok{lbph=}\FloatTok{0.30010}\NormalTok{,}
                                              \AttributeTok{svi=}\DecValTok{0}\NormalTok{,}
                                              \AttributeTok{lcp=}\SpecialCharTok{{-}}\FloatTok{0.79851}\NormalTok{,}
                                              \AttributeTok{gleason=}\DecValTok{7}\NormalTok{,}
                                              \AttributeTok{pgg45=}\DecValTok{15}\NormalTok{))}
\end{Highlighting}
\end{Shaded}

\begin{verbatim}
##    lcavol lweight age   lbph svi      lcp gleason pgg45
## 1 1.44692 3.62301  65 0.3001   0 -0.79851       7    15
\end{verbatim}

\begin{Shaded}
\begin{Highlighting}[]
\FunctionTok{predict}\NormalTok{(lmod5, x0pros, }\AttributeTok{interval=}\StringTok{"prediction"}\NormalTok{, }\AttributeTok{level=}\FloatTok{0.95}\NormalTok{)}
\end{Highlighting}
\end{Shaded}

\begin{verbatim}
##        fit       lwr      upr
## 1 2.389053 0.9646584 3.813447
\end{verbatim}

\hypertarget{b-repeat-the-last-question-for-a-patient-with-the-same-values-except-that-he-is-age-20.-explain-why-the-ci-is-wider.}{%
\paragraph{(b) Repeat the last question for a patient with the same
values except that he is age 20. Explain why the CI is
wider.}\label{b-repeat-the-last-question-for-a-patient-with-the-same-values-except-that-he-is-age-20.-explain-why-the-ci-is-wider.}}

\begin{Shaded}
\begin{Highlighting}[]
\NormalTok{x1pros }\OtherTok{\textless{}{-}} \FunctionTok{data.frame}\NormalTok{(}\AttributeTok{lcavol=}\FloatTok{1.44692}\NormalTok{,}\AttributeTok{lweight=}\FloatTok{3.62301}\NormalTok{,}\AttributeTok{age=}\DecValTok{20}\NormalTok{,}\AttributeTok{lbph=}\FloatTok{0.30010}\NormalTok{,}\AttributeTok{svi=}\DecValTok{0}\NormalTok{,}\AttributeTok{lcp=}\SpecialCharTok{{-}}\FloatTok{0.79851}\NormalTok{,}\AttributeTok{gleason=}\DecValTok{7}\NormalTok{,}\AttributeTok{pgg45=}\DecValTok{15}\NormalTok{)}
\FunctionTok{predict}\NormalTok{(lmod5,x1pros,}\AttributeTok{interval=}\StringTok{"prediction"}\NormalTok{)}
\end{Highlighting}
\end{Shaded}

\begin{verbatim}
##        fit      lwr      upr
## 1 3.272726 1.538744 5.006707
\end{verbatim}

E el apartado B, el valor para la edad, está fuera del rango de los
valores del dataset, por eso el intervalo de confianza es mayor.

\hypertarget{c-for-the-model-of-the-previous-question-remove-all-the-predictors-that-are-not-significant-at-the-5-level.-now-recompute-the-predictions-of-the-previous-question.-are-the-cis-wider-or-narrower-which-predictions-would-you-prefer-explain.}{%
\paragraph{(c) For the model of the previous question, remove all the
predictors that are not significant at the 5\% level. Now recompute the
predictions of the previous question. Are the CIs wider or narrower?
Which predictions would you prefer?
Explain.}\label{c-for-the-model-of-the-previous-question-remove-all-the-predictors-that-are-not-significant-at-the-5-level.-now-recompute-the-predictions-of-the-previous-question.-are-the-cis-wider-or-narrower-which-predictions-would-you-prefer-explain.}}

\begin{Shaded}
\begin{Highlighting}[]
 \FunctionTok{summary}\NormalTok{(lmod5)}
\end{Highlighting}
\end{Shaded}

\begin{verbatim}
## 
## Call:
## lm(formula = lpsa ~ lcavol + lweight + age + lbph + svi + lcp + 
##     gleason + pgg45, data = prostate)
## 
## Residuals:
##     Min      1Q  Median      3Q     Max 
## -1.7331 -0.3713 -0.0170  0.4141  1.6381 
## 
## Coefficients:
##              Estimate Std. Error t value Pr(>|t|)    
## (Intercept)  0.669337   1.296387   0.516  0.60693    
## lcavol       0.587022   0.087920   6.677 2.11e-09 ***
## lweight      0.454467   0.170012   2.673  0.00896 ** 
## age         -0.019637   0.011173  -1.758  0.08229 .  
## lbph         0.107054   0.058449   1.832  0.07040 .  
## svi          0.766157   0.244309   3.136  0.00233 ** 
## lcp         -0.105474   0.091013  -1.159  0.24964    
## gleason      0.045142   0.157465   0.287  0.77503    
## pgg45        0.004525   0.004421   1.024  0.30886    
## ---
## Signif. codes:  0 '***' 0.001 '**' 0.01 '*' 0.05 '.' 0.1 ' ' 1
## 
## Residual standard error: 0.7084 on 88 degrees of freedom
## Multiple R-squared:  0.6548, Adjusted R-squared:  0.6234 
## F-statistic: 20.86 on 8 and 88 DF,  p-value: < 2.2e-16
\end{verbatim}

Las variables con significación al 5\% son ``lcavol'', ``lweight'',
``svi''. Creamos un nuevo modelo con ellas.

\begin{Shaded}
\begin{Highlighting}[]
\NormalTok{lmod6}\OtherTok{\textless{}{-}}\FunctionTok{lm}\NormalTok{(lpsa }\SpecialCharTok{\textasciitilde{}}\NormalTok{ lcavol }\SpecialCharTok{+}\NormalTok{ lweight }\SpecialCharTok{+}\NormalTok{ svi, }\AttributeTok{data=}\NormalTok{prostate)}
\FunctionTok{summary}\NormalTok{(lmod6)}
\end{Highlighting}
\end{Shaded}

\begin{verbatim}
## 
## Call:
## lm(formula = lpsa ~ lcavol + lweight + svi, data = prostate)
## 
## Residuals:
##      Min       1Q   Median       3Q      Max 
## -1.72964 -0.45764  0.02812  0.46403  1.57013 
## 
## Coefficients:
##             Estimate Std. Error t value Pr(>|t|)    
## (Intercept) -0.26809    0.54350  -0.493  0.62298    
## lcavol       0.55164    0.07467   7.388  6.3e-11 ***
## lweight      0.50854    0.15017   3.386  0.00104 ** 
## svi          0.66616    0.20978   3.176  0.00203 ** 
## ---
## Signif. codes:  0 '***' 0.001 '**' 0.01 '*' 0.05 '.' 0.1 ' ' 1
## 
## Residual standard error: 0.7168 on 93 degrees of freedom
## Multiple R-squared:  0.6264, Adjusted R-squared:  0.6144 
## F-statistic: 51.99 on 3 and 93 DF,  p-value: < 2.2e-16
\end{verbatim}

\begin{Shaded}
\begin{Highlighting}[]
\FunctionTok{predict}\NormalTok{(lmod6, x0pros, }\AttributeTok{interval=}\StringTok{"prediction"}\NormalTok{)}
\end{Highlighting}
\end{Shaded}

\begin{verbatim}
##        fit       lwr      upr
## 1 2.372534 0.9383436 3.806724
\end{verbatim}

\begin{Shaded}
\begin{Highlighting}[]
\FunctionTok{predict}\NormalTok{(lmod6, x1pros, }\AttributeTok{interval=}\StringTok{"prediction"}\NormalTok{)}
\end{Highlighting}
\end{Shaded}

\begin{verbatim}
##        fit       lwr      upr
## 1 2.372534 0.9383436 3.806724
\end{verbatim}

\hypertarget{ejercicio-2-cap.-4-puxe1g.-57}{%
\paragraph{2. (Ejercicio 2 cap. 4 pág.
57)}\label{ejercicio-2-cap.-4-puxe1g.-57}}

Using the teengamb data, fit a model with gamble as the response and the
other variables as predictors.

\begin{Shaded}
\begin{Highlighting}[]
\FunctionTok{data}\NormalTok{(teengamb)}
\NormalTok{lmod7}\OtherTok{\textless{}{-}}\FunctionTok{lm}\NormalTok{(gamble }\SpecialCharTok{\textasciitilde{}}\NormalTok{ sex }\SpecialCharTok{+}\NormalTok{ status }\SpecialCharTok{+}\NormalTok{ income }\SpecialCharTok{+}\NormalTok{verbal, }\AttributeTok{data=}\NormalTok{teengamb)}
\FunctionTok{summary}\NormalTok{(lmod7)}
\end{Highlighting}
\end{Shaded}

\begin{verbatim}
## 
## Call:
## lm(formula = gamble ~ sex + status + income + verbal, data = teengamb)
## 
## Residuals:
##     Min      1Q  Median      3Q     Max 
## -51.082 -11.320  -1.451   9.452  94.252 
## 
## Coefficients:
##              Estimate Std. Error t value Pr(>|t|)    
## (Intercept)  22.55565   17.19680   1.312   0.1968    
## sex         -22.11833    8.21111  -2.694   0.0101 *  
## status        0.05223    0.28111   0.186   0.8535    
## income        4.96198    1.02539   4.839 1.79e-05 ***
## verbal       -2.95949    2.17215  -1.362   0.1803    
## ---
## Signif. codes:  0 '***' 0.001 '**' 0.01 '*' 0.05 '.' 0.1 ' ' 1
## 
## Residual standard error: 22.69 on 42 degrees of freedom
## Multiple R-squared:  0.5267, Adjusted R-squared:  0.4816 
## F-statistic: 11.69 on 4 and 42 DF,  p-value: 1.815e-06
\end{verbatim}

\hypertarget{a-predict-the-amount-that-a-male-with-average-given-these-data-status-income-and-verbal-score-would-gamble-along-with-an-appropriate-95-ci.}{%
\paragraph{(a) Predict the amount that a male with average (given these
data) status, income and verbal score would gamble along with an
appropriate 95\%
CI.}\label{a-predict-the-amount-that-a-male-with-average-given-these-data-status-income-and-verbal-score-would-gamble-along-with-an-appropriate-95-ci.}}

\begin{Shaded}
\begin{Highlighting}[]
\FunctionTok{attach}\NormalTok{(teengamb)}
\NormalTok{x0gamb}\OtherTok{\textless{}{-}}\FunctionTok{data.frame}\NormalTok{( }\AttributeTok{sex=}\DecValTok{0}\NormalTok{, }\AttributeTok{status=}\FunctionTok{median}\NormalTok{(status), }
                                  \AttributeTok{income=}\FunctionTok{median}\NormalTok{(income), }\AttributeTok{verbal=}\FunctionTok{median}\NormalTok{(verbal))}

\FunctionTok{predict}\NormalTok{(lmod7,x0gamb,}\AttributeTok{interval=}\StringTok{"prediction"}\NormalTok{)}
\end{Highlighting}
\end{Shaded}

\begin{verbatim}
##        fit       lwr      upr
## 1 20.21168 -26.98701 67.41037
\end{verbatim}

\hypertarget{b-repeat-the-prediction-for-a-male-with-maximal-values-for-this-data-of-status-income-and-verbal-score.-which-ci-is-wider-and-why-is-this-result-expected}{%
\paragraph{(b) Repeat the prediction for a male with maximal values (for
this data) of status, income and verbal score. Which CI is wider and why
is this result
expected?}\label{b-repeat-the-prediction-for-a-male-with-maximal-values-for-this-data-of-status-income-and-verbal-score.-which-ci-is-wider-and-why-is-this-result-expected}}

\begin{Shaded}
\begin{Highlighting}[]
\NormalTok{x1gamb }\OtherTok{\textless{}{-}} \FunctionTok{data.frame}\NormalTok{(}\AttributeTok{sex=}\DecValTok{0}\NormalTok{, }\AttributeTok{status=}\FunctionTok{max}\NormalTok{(status),}
                                            \AttributeTok{income=}\FunctionTok{max}\NormalTok{(income),}\AttributeTok{verbal=}\FunctionTok{max}\NormalTok{(verbal))}

\FunctionTok{predict}\NormalTok{(lmod7,x1gamb,}\AttributeTok{interval=}\StringTok{"prediction"}\NormalTok{)}
\end{Highlighting}
\end{Shaded}

\begin{verbatim}
##        fit      lwr    upr
## 1 71.30794 17.06588 125.55
\end{verbatim}

\hypertarget{c-fit-a-model-with-sqrtgamble-as-the-response-but-with-the-same-predictors.-now-predict-the-response-and-give-a-95-prediction-interval-for-the-individual-in-a.-take-care-to-give-your-answer-in-the-original-units-of-the-response.}{%
\paragraph{(c) Fit a model with sqrt(gamble) as the response but with
the same predictors. Now predict the response and give a 95\% prediction
interval for the individual in (a). Take care to give your answer in the
original units of the
response.}\label{c-fit-a-model-with-sqrtgamble-as-the-response-but-with-the-same-predictors.-now-predict-the-response-and-give-a-95-prediction-interval-for-the-individual-in-a.-take-care-to-give-your-answer-in-the-original-units-of-the-response.}}

\begin{Shaded}
\begin{Highlighting}[]
\NormalTok{lmod8}\OtherTok{\textless{}{-}}\FunctionTok{lm}\NormalTok{(}\FunctionTok{sqrt}\NormalTok{(gamble) }\SpecialCharTok{\textasciitilde{}}\NormalTok{ sex }\SpecialCharTok{+}\NormalTok{ status }\SpecialCharTok{+}\NormalTok{ income }\SpecialCharTok{+}\NormalTok{verbal, }\AttributeTok{data=}\NormalTok{teengamb)}
\FunctionTok{predict}\NormalTok{(lmod8,x0gamb,}\AttributeTok{interval=}\StringTok{"prediction"}\NormalTok{)}
\end{Highlighting}
\end{Shaded}

\begin{verbatim}
##        fit       lwr      upr
## 1 3.155673 -1.179373 7.490718
\end{verbatim}

\begin{Shaded}
\begin{Highlighting}[]
\FunctionTok{predict}\NormalTok{(lmod8,x0gamb,}\AttributeTok{interval=}\StringTok{"prediction"}\NormalTok{)}\SpecialCharTok{\^{}}\DecValTok{2}
\end{Highlighting}
\end{Shaded}

\begin{verbatim}
##        fit     lwr      upr
## 1 9.958271 1.39092 56.11086
\end{verbatim}

\hypertarget{d-repeat-the-prediction-for-the-model-in-c-for-a-female-with-status-20-income-1-verbal-10.-comment-on-the-credibility-of-the-result}{%
\paragraph{(d) Repeat the prediction for the model in (c) for a female
with status = 20, income = 1, verbal = 10. Comment on the credibility of
the
result}\label{d-repeat-the-prediction-for-the-model-in-c-for-a-female-with-status-20-income-1-verbal-10.-comment-on-the-credibility-of-the-result}}

\begin{Shaded}
\begin{Highlighting}[]
\NormalTok{x2gamb }\OtherTok{\textless{}{-}} \FunctionTok{data.frame}\NormalTok{(}\AttributeTok{sex=}\DecValTok{1}\NormalTok{,}\AttributeTok{status=}\DecValTok{20}\NormalTok{,}\AttributeTok{income=}\DecValTok{1}\NormalTok{,}\AttributeTok{verbal=}\DecValTok{10}\NormalTok{)}
 \FunctionTok{predict}\NormalTok{(lmod8,x2gamb,}\AttributeTok{interval=}\StringTok{"prediction"}\NormalTok{)}\SpecialCharTok{\^{}}\DecValTok{2}
\end{Highlighting}
\end{Shaded}

\begin{verbatim}
##        fit      lwr      upr
## 1 4.353398 47.73238 7.485167
\end{verbatim}

\hypertarget{ejercicio-3-cap.-4-puxe1g.-57}{%
\subsection{3. (Ejercicio 3 cap. 4 pág.
57)}\label{ejercicio-3-cap.-4-puxe1g.-57}}

The snail dataset contains percentage water content of the tissues of
snails grown under three different levels of relative humidity and two
different temperatures.

\hypertarget{a-use-the-command-xtabswater-temp-humid-snail4-to-produce-a-table-of-mean-water-content-for-each-combination-of-temperature-and-humidity.-can-you-use-this-table-to-predict-the-water-content-for-a-temperature-of-25-c-and-a-humidity-of-60-explain.}{%
\paragraph{(a) Use the command xtabs(water \textasciitilde{} temp +
humid, snail)/4 to produce a table of mean water content for each
combination of temperature and humidity. Can you use this table to
predict the water content for a temperature of 25 ◦ C and a humidity of
60\%?
Explain.}\label{a-use-the-command-xtabswater-temp-humid-snail4-to-produce-a-table-of-mean-water-content-for-each-combination-of-temperature-and-humidity.-can-you-use-this-table-to-predict-the-water-content-for-a-temperature-of-25-c-and-a-humidity-of-60-explain.}}

\begin{Shaded}
\begin{Highlighting}[]
\FunctionTok{require}\NormalTok{(faraway)}
\FunctionTok{data}\NormalTok{(snail)}
\end{Highlighting}
\end{Shaded}

\begin{Shaded}
\begin{Highlighting}[]
\FunctionTok{xtabs}\NormalTok{(water }\SpecialCharTok{\textasciitilde{}}\NormalTok{ temp }\SpecialCharTok{+}\NormalTok{ humid, snail)}\SpecialCharTok{/}\DecValTok{4}
\end{Highlighting}
\end{Shaded}

\begin{verbatim}
##     humid
## temp    45    75   100
##   20 72.50 81.50 97.00
##   30 69.50 78.25 97.75
\end{verbatim}

\begin{Shaded}
\begin{Highlighting}[]
\NormalTok{mytable }\OtherTok{\textless{}{-}} \FunctionTok{xtabs}\NormalTok{(water}\SpecialCharTok{\textasciitilde{}}\NormalTok{temp}\SpecialCharTok{+}\NormalTok{humid,snail)}\SpecialCharTok{/}\DecValTok{4}
\FunctionTok{colnames}\NormalTok{(mytable) }\OtherTok{\textless{}{-}} \FunctionTok{c}\NormalTok{(}\DecValTok{45}\NormalTok{,}\DecValTok{75}\NormalTok{,}\DecValTok{100}\NormalTok{)}
\FunctionTok{rownames}\NormalTok{(mytable) }\OtherTok{\textless{}{-}} \FunctionTok{c}\NormalTok{(}\DecValTok{20}\NormalTok{, }\DecValTok{30}\NormalTok{)}
\FunctionTok{matplot}\NormalTok{ (}\FunctionTok{t}\NormalTok{(mytable), }\AttributeTok{type=}\StringTok{"l"}\NormalTok{)}
\end{Highlighting}
\end{Shaded}

\includegraphics{Regresion-simple-y-multiple_files/figure-latex/unnamed-chunk-40-1.pdf}

\begin{Shaded}
\begin{Highlighting}[]
\FunctionTok{xtabs}\NormalTok{(}\FunctionTok{mean}\NormalTok{(water[humid}\SpecialCharTok{\textless{}}\DecValTok{100}\NormalTok{])}\SpecialCharTok{\textasciitilde{}}\FunctionTok{mean}\NormalTok{(temp[humid}\SpecialCharTok{\textless{}}\DecValTok{100}\NormalTok{])}\SpecialCharTok{+}\FunctionTok{mean}\NormalTok{(humid[humid}\SpecialCharTok{\textless{}}\DecValTok{100}\NormalTok{]),snail)}
\end{Highlighting}
\end{Shaded}

\begin{verbatim}
##                        mean(humid[humid < 100])
## mean(temp[humid < 100])      60
##                      25 75.4375
\end{verbatim}

\hypertarget{b-fit-a-regression-model-with-the-water-content-as-the-response-and-the-temperature-and-humidity-as-predictors.-use-this-model-to-predict-the-water-content-for-a-temperature-of-25-c-and-a-humidity-of-60}{%
\paragraph{(b) Fit a regression model with the water content as the
response and the temperature and humidity as predictors. Use this model
to predict the water content for a temperature of 25 ◦ C and a humidity
of
60\%?}\label{b-fit-a-regression-model-with-the-water-content-as-the-response-and-the-temperature-and-humidity-as-predictors.-use-this-model-to-predict-the-water-content-for-a-temperature-of-25-c-and-a-humidity-of-60}}

\begin{Shaded}
\begin{Highlighting}[]
\NormalTok{lmod9}\OtherTok{\textless{}{-}}\NormalTok{(}\FunctionTok{lm}\NormalTok{(water }\SpecialCharTok{\textasciitilde{}}\NormalTok{ temp }\SpecialCharTok{+}\NormalTok{ humid, }\AttributeTok{data=}\NormalTok{snail))}
\FunctionTok{summary}\NormalTok{(lmod9)}
\end{Highlighting}
\end{Shaded}

\begin{verbatim}
## 
## Call:
## lm(formula = water ~ temp + humid, data = snail)
## 
## Residuals:
##     Min      1Q  Median      3Q     Max 
## -12.456  -2.915   1.461   3.613   8.749 
## 
## Coefficients:
##             Estimate Std. Error t value Pr(>|t|)    
## (Intercept) 52.61081    6.85346   7.677 1.59e-07 ***
## temp        -0.18333    0.22645  -0.810    0.427    
## humid        0.47349    0.05036   9.403 5.63e-09 ***
## ---
## Signif. codes:  0 '***' 0.001 '**' 0.01 '*' 0.05 '.' 0.1 ' ' 1
## 
## Residual standard error: 5.547 on 21 degrees of freedom
## Multiple R-squared:  0.8092, Adjusted R-squared:  0.791 
## F-statistic: 44.53 on 2 and 21 DF,  p-value: 2.793e-08
\end{verbatim}

\begin{Shaded}
\begin{Highlighting}[]
\NormalTok{x0wat}\OtherTok{\textless{}{-}}\FunctionTok{data.frame}\NormalTok{(}\AttributeTok{temp=}\DecValTok{25}\NormalTok{,}\AttributeTok{humid=}\DecValTok{60}\NormalTok{)}
\FunctionTok{predict}\NormalTok{(lmod9, x0wat, }\AttributeTok{interval=}\StringTok{"prediction"}\NormalTok{)}
\end{Highlighting}
\end{Shaded}

\begin{verbatim}
##        fit      lwr      upr
## 1 76.43681 64.58094 88.29269
\end{verbatim}

\hypertarget{c-use-this-model-to-predict-water-content-for-a-temperature-of-30-c-and-a-humidity-of-75.-compare-your-prediction-to-the-prediction-from-a.-discuss-the-relative-merits-of-these-two-predictions.}{%
\paragraph{(c) Use this model to predict water content for a temperature
of 30 ◦ C and a humidity of 75\%?. Compare your prediction to the
prediction from (a). Discuss the relative merits of these two
predictions.}\label{c-use-this-model-to-predict-water-content-for-a-temperature-of-30-c-and-a-humidity-of-75.-compare-your-prediction-to-the-prediction-from-a.-discuss-the-relative-merits-of-these-two-predictions.}}

\begin{Shaded}
\begin{Highlighting}[]
\NormalTok{x1wat}\OtherTok{\textless{}{-}}\FunctionTok{data.frame}\NormalTok{(}\AttributeTok{temp=}\DecValTok{30}\NormalTok{,}\AttributeTok{humid=}\DecValTok{75}\NormalTok{)}
\FunctionTok{predict}\NormalTok{(lmod9, x1wat, }\AttributeTok{interval=}\StringTok{"prediction"}\NormalTok{)}
\end{Highlighting}
\end{Shaded}

\begin{verbatim}
##        fit     lwr      upr
## 1 82.62248 70.6147 94.63027
\end{verbatim}

\hypertarget{d-the-intercept-in-your-model-is-52.6.-give-two-values-of-the-predictors-for-which-this-represents-the-predicted-response.-is-your-answer-unique-do-you-think-this-represents-a-reasonable-prediction}{%
\paragraph{(d) The intercept in your model is 52.6\%. Give two values of
the predictors for which this represents the predicted response. Is your
answer unique? Do you think this represents a reasonable
prediction?}\label{d-the-intercept-in-your-model-is-52.6.-give-two-values-of-the-predictors-for-which-this-represents-the-predicted-response.-is-your-answer-unique-do-you-think-this-represents-a-reasonable-prediction}}

\begin{Shaded}
\begin{Highlighting}[]
\NormalTok{x3wat}\OtherTok{\textless{}{-}}\FunctionTok{data.frame}\NormalTok{(}\AttributeTok{temp=}\DecValTok{0}\NormalTok{,}\AttributeTok{humid=}\DecValTok{0}\NormalTok{)}
\FunctionTok{predict}\NormalTok{(lmod9,x3wat,}\AttributeTok{interval=}\StringTok{"prediction"}\NormalTok{)}
\end{Highlighting}
\end{Shaded}

\begin{verbatim}
##        fit      lwr      upr
## 1 52.61081 34.27498 70.94663
\end{verbatim}

\hypertarget{e-for-a-temperature-of-25-c-what-value-of-humidity-would-give-a-predicted-response-of-80-water-content.}{%
\paragraph{(e) For a temperature of 25 ◦ C, what value of humidity would
give a predicted response of 80\% water
content.}\label{e-for-a-temperature-of-25-c-what-value-of-humidity-would-give-a-predicted-response-of-80-water-content.}}

\begin{Shaded}
\begin{Highlighting}[]
\NormalTok{x6wat}\OtherTok{\textless{}{-}}\FunctionTok{data.frame}\NormalTok{(}\AttributeTok{temp=}\DecValTok{25}\NormalTok{,}\AttributeTok{humid=}\FunctionTok{seq}\NormalTok{(}\FloatTok{67.1}\NormalTok{,}\FloatTok{68.0}\NormalTok{,}\FloatTok{0.1}\NormalTok{))}
\FunctionTok{predict}\NormalTok{(lmod9,x6wat,}\AttributeTok{interval=}\StringTok{"prediction"}\NormalTok{)}
\end{Highlighting}
\end{Shaded}

\begin{verbatim}
##         fit      lwr      upr
## 1  79.79859 68.00714 91.59003
## 2  79.84593 68.05507 91.63680
## 3  79.89328 68.10298 91.68358
## 4  79.94063 68.15089 91.73038
## 5  79.98798 68.19878 91.77718
## 6  80.03533 68.24667 91.82399
## 7  80.08268 68.29455 91.87081
## 8  80.13003 68.34242 91.91764
## 9  80.17738 68.39028 91.96448
## 10 80.22473 68.43813 92.01133
\end{verbatim}

\hypertarget{ejercicios-opcionales}{%
\subsection{Ejercicios Opcionales}\label{ejercicios-opcionales}}

\hypertarget{ejercicios-del-libro-de-carmona}{%
\subsection{Ejercicios del libro de
Carmona}\label{ejercicios-del-libro-de-carmona}}

\hypertarget{ejercicio-6.9-del-capuxedtulo-6-puxe1gina-118}{%
\paragraph{2. (∗) (Ejercicio 6.9 del Capítulo 6 página
118)}\label{ejercicio-6.9-del-capuxedtulo-6-puxe1gina-118}}

Comparar las rectas de regresión de hombres y mujeres con los logaritmos
de los datos del ejercicio 1.4.

\begin{Shaded}
\begin{Highlighting}[]
\CommentTok{\# Datos}
\NormalTok{TPO\_H }\OtherTok{\textless{}{-}} \FunctionTok{c}\NormalTok{(}\FloatTok{9.84}\NormalTok{,}\FloatTok{19.32}\NormalTok{,}\FloatTok{43.19}\NormalTok{,}\FloatTok{102.58}\NormalTok{,}\FloatTok{215.78}\NormalTok{,}\FloatTok{787.96}\NormalTok{,}\FloatTok{1627.34}\NormalTok{,}\DecValTok{7956}\NormalTok{)}
\NormalTok{TPO\_M }\OtherTok{\textless{}{-}} \FunctionTok{c}\NormalTok{(}\FloatTok{10.94}\NormalTok{,}\FloatTok{22.12}\NormalTok{,}\FloatTok{48.25}\NormalTok{,}\FloatTok{117.73}\NormalTok{,}\FloatTok{240.83}\NormalTok{,}\FloatTok{899.88}\NormalTok{,}\FloatTok{1861.63}\NormalTok{,}\DecValTok{8765}\NormalTok{)}
\NormalTok{distancia }\OtherTok{\textless{}{-}} \FunctionTok{c}\NormalTok{(}\DecValTok{100}\NormalTok{,}\DecValTok{200}\NormalTok{,}\DecValTok{400}\NormalTok{,}\DecValTok{800}\NormalTok{,}\DecValTok{1500}\NormalTok{,}\DecValTok{5000}\NormalTok{,}\DecValTok{10000}\NormalTok{,}\DecValTok{42192}\NormalTok{)}
\NormalTok{lTPO\_H }\OtherTok{\textless{}{-}} \FunctionTok{log}\NormalTok{(TPO\_H)}
\NormalTok{lTPO\_M }\OtherTok{\textless{}{-}} \FunctionTok{log}\NormalTok{(TPO\_M)}
\NormalTok{ldistancia }\OtherTok{\textless{}{-}} \FunctionTok{log}\NormalTok{(distancia)}
\end{Highlighting}
\end{Shaded}

\begin{Shaded}
\begin{Highlighting}[]
\NormalTok{n }\OtherTok{\textless{}{-}} \FunctionTok{length}\NormalTok{(distancia)}
\NormalTok{uno.h }\OtherTok{\textless{}{-}} \FunctionTok{c}\NormalTok{(}\FunctionTok{rep}\NormalTok{(}\DecValTok{1}\NormalTok{,n),}\FunctionTok{rep}\NormalTok{(}\DecValTok{0}\NormalTok{,n))}
\NormalTok{uno.m }\OtherTok{\textless{}{-}} \FunctionTok{c}\NormalTok{(}\FunctionTok{rep}\NormalTok{(}\DecValTok{0}\NormalTok{,n),}\FunctionTok{rep}\NormalTok{(}\DecValTok{1}\NormalTok{,n))}
\NormalTok{x.h }\OtherTok{\textless{}{-}} \FunctionTok{c}\NormalTok{(ldistancia,}\FunctionTok{rep}\NormalTok{(}\DecValTok{0}\NormalTok{,n))}
\NormalTok{x.m }\OtherTok{\textless{}{-}} \FunctionTok{c}\NormalTok{(}\FunctionTok{rep}\NormalTok{(}\DecValTok{0}\NormalTok{,n),ldistancia)}
\NormalTok{y }\OtherTok{\textless{}{-}} \FunctionTok{c}\NormalTok{(lTPO\_H,lTPO\_M)}
\NormalTok{modc }\OtherTok{\textless{}{-}} \FunctionTok{lm}\NormalTok{(y }\SpecialCharTok{\textasciitilde{}} \DecValTok{0} \SpecialCharTok{+}\NormalTok{ uno.h }\SpecialCharTok{+}\NormalTok{ uno.m }\SpecialCharTok{+}\NormalTok{ x.h }\SpecialCharTok{+}\NormalTok{ x.m)}
\end{Highlighting}
\end{Shaded}

\begin{Shaded}
\begin{Highlighting}[]
\NormalTok{x }\OtherTok{\textless{}{-}} \FunctionTok{c}\NormalTok{(ldistancia,ldistancia)}
\NormalTok{modp }\OtherTok{\textless{}{-}} \FunctionTok{lm}\NormalTok{(y }\SpecialCharTok{\textasciitilde{}} \DecValTok{0} \SpecialCharTok{+}\NormalTok{ uno.h }\SpecialCharTok{+}\NormalTok{ uno.m }\SpecialCharTok{+}\NormalTok{ x)}
\end{Highlighting}
\end{Shaded}

\begin{Shaded}
\begin{Highlighting}[]
\NormalTok{mod0 }\OtherTok{\textless{}{-}} \FunctionTok{lm}\NormalTok{(y }\SpecialCharTok{\textasciitilde{}}\NormalTok{ x)}
\FunctionTok{anova}\NormalTok{(mod0,modp)}
\end{Highlighting}
\end{Shaded}

\begin{verbatim}
## Analysis of Variance Table
## 
## Model 1: y ~ x
## Model 2: y ~ 0 + uno.h + uno.m + x
##   Res.Df      RSS Df Sum of Sq     F   Pr(>F)   
## 1     14 0.100610                               
## 2     13 0.042548  1  0.058062 17.74 0.001017 **
## ---
## Signif. codes:  0 '***' 0.001 '**' 0.01 '*' 0.05 '.' 0.1 ' ' 1
\end{verbatim}

\hypertarget{ejercicio-8.5-del-capuxedtulo-8-puxe1gina-157}{%
\paragraph{6. (∗) (Ejercicio 8.5 del Capítulo 8 página
157)}\label{ejercicio-8.5-del-capuxedtulo-8-puxe1gina-157}}

Dado el modelo Y t = β 0 + β 1 X 1t + β 2 X 2t + u t y los siguientes
datos:

\begin{Shaded}
\begin{Highlighting}[]
\NormalTok{y }\OtherTok{\textless{}{-}} \FunctionTok{c}\NormalTok{(}\DecValTok{10}\NormalTok{,}\DecValTok{25}\NormalTok{,}\DecValTok{32}\NormalTok{,}\DecValTok{43}\NormalTok{,}\DecValTok{58}\NormalTok{,}\DecValTok{62}\NormalTok{,}\DecValTok{67}\NormalTok{,}\DecValTok{71}\NormalTok{)}
\NormalTok{x1 }\OtherTok{\textless{}{-}} \FunctionTok{c}\NormalTok{(}\DecValTok{1}\NormalTok{,}\DecValTok{3}\NormalTok{,}\DecValTok{4}\NormalTok{,}\DecValTok{5}\NormalTok{,}\DecValTok{7}\NormalTok{,}\DecValTok{8}\NormalTok{,}\DecValTok{10}\NormalTok{,}\DecValTok{10}\NormalTok{)}
\NormalTok{x2 }\OtherTok{\textless{}{-}} \FunctionTok{c}\NormalTok{(}\DecValTok{0}\NormalTok{,}\SpecialCharTok{{-}}\DecValTok{1}\NormalTok{,}\DecValTok{0}\NormalTok{,}\DecValTok{1}\NormalTok{,}\SpecialCharTok{{-}}\DecValTok{1}\NormalTok{,}\DecValTok{0}\NormalTok{,}\SpecialCharTok{{-}}\DecValTok{1}\NormalTok{,}\DecValTok{2}\NormalTok{)}
\end{Highlighting}
\end{Shaded}

\hypertarget{a-la-estimaciuxf3n-mc-de-ux3b2-0-ux3b2-1-ux3b2-2-utilizando-los-valores-originales.uxe7}{%
\paragraph{(a) La estimación MC de β 0 , β 1 , β 2 utilizando los
valores
originales.ç}\label{a-la-estimaciuxf3n-mc-de-ux3b2-0-ux3b2-1-ux3b2-2-utilizando-los-valores-originales.uxe7}}

\begin{Shaded}
\begin{Highlighting}[]
\NormalTok{lmod10}\OtherTok{\textless{}{-}}\FunctionTok{lm}\NormalTok{(y }\SpecialCharTok{\textasciitilde{}}\NormalTok{ x1 }\SpecialCharTok{+}\NormalTok{ x2)}
\FunctionTok{coef}\NormalTok{(lmod10)}
\end{Highlighting}
\end{Shaded}

\begin{verbatim}
## (Intercept)          x1          x2 
##   6.4699828   6.5883362   0.2572899
\end{verbatim}

\hypertarget{b-la-estimaciuxf3n-mc-de-ux3b2-0-ux3b2-1-ux3b2-2-utilizando-los-datos-expresados-en-desviaciones-respecto-de-la-media.}{%
\paragraph{(b) La estimación MC de β 0 , β 1 , β 2 utilizando los datos
expresados en desviaciones respecto de la
media.}\label{b-la-estimaciuxf3n-mc-de-ux3b2-0-ux3b2-1-ux3b2-2-utilizando-los-datos-expresados-en-desviaciones-respecto-de-la-media.}}

\begin{Shaded}
\begin{Highlighting}[]
\NormalTok{ys }\OtherTok{\textless{}{-}} \FunctionTok{scale}\NormalTok{(y, }\AttributeTok{center=}\ConstantTok{TRUE}\NormalTok{, }\AttributeTok{scale=}\ConstantTok{FALSE}\NormalTok{)}
\NormalTok{x1s }\OtherTok{\textless{}{-}} \FunctionTok{scale}\NormalTok{(x1, }\AttributeTok{center=}\ConstantTok{TRUE}\NormalTok{, }\AttributeTok{scale=}\ConstantTok{FALSE}\NormalTok{)}
\NormalTok{x2s }\OtherTok{\textless{}{-}} \FunctionTok{scale}\NormalTok{(x2, }\AttributeTok{center=}\ConstantTok{TRUE}\NormalTok{, }\AttributeTok{scale=}\ConstantTok{FALSE}\NormalTok{)}
\NormalTok{lmod11 }\OtherTok{\textless{}{-}} \FunctionTok{lm}\NormalTok{(ys }\SpecialCharTok{\textasciitilde{}} \DecValTok{0} \SpecialCharTok{+}\NormalTok{ x1s }\SpecialCharTok{+}\NormalTok{ x2s)}
\FunctionTok{coef}\NormalTok{(lmod11)}
\end{Highlighting}
\end{Shaded}

\begin{verbatim}
##       x1s       x2s 
## 6.5883362 0.2572899
\end{verbatim}

\hypertarget{c-la-estimaciuxf3n-insesgada-de-ux3c32}{%
\paragraph{(c) La estimación insesgada de
σ2}\label{c-la-estimaciuxf3n-insesgada-de-ux3c32}}

\begin{Shaded}
\begin{Highlighting}[]
\NormalTok{sg }\OtherTok{\textless{}{-}} \FunctionTok{summary}\NormalTok{(lmod10)}
\NormalTok{sgs }\OtherTok{\textless{}{-}} \FunctionTok{summary}\NormalTok{(lmod11)}
\FunctionTok{c}\NormalTok{(sg}\SpecialCharTok{$}\NormalTok{sigma}\SpecialCharTok{\^{}}\DecValTok{2}\NormalTok{, sgs}\SpecialCharTok{$}\NormalTok{sigma}\SpecialCharTok{\^{}}\DecValTok{2}\NormalTok{)}
\end{Highlighting}
\end{Shaded}

\begin{verbatim}
## [1] 18.33002 15.27501
\end{verbatim}

\hypertarget{d-el-coeficiente-de-determinaciuxf3n}{%
\paragraph{(d) El coeficiente de
determinación}\label{d-el-coeficiente-de-determinaciuxf3n}}

\begin{Shaded}
\begin{Highlighting}[]
\FunctionTok{c}\NormalTok{(sg}\SpecialCharTok{$}\NormalTok{r.squared,sgs}\SpecialCharTok{$}\NormalTok{r.squared)}
\end{Highlighting}
\end{Shaded}

\begin{verbatim}
## [1] 0.9731074 0.9731074
\end{verbatim}

\hypertarget{e-el-coeficiente-de-determinaciuxf3n-corregido}{%
\paragraph{(e) El coeficiente de determinación
corregido}\label{e-el-coeficiente-de-determinaciuxf3n-corregido}}

\begin{Shaded}
\begin{Highlighting}[]
\FunctionTok{c}\NormalTok{(sg}\SpecialCharTok{$}\NormalTok{adj.r.squared,sgs}\SpecialCharTok{$}\NormalTok{adj.r.squared)}
\end{Highlighting}
\end{Shaded}

\begin{verbatim}
## [1] 0.9623503 0.9641432
\end{verbatim}

\hypertarget{f-el-contraste-de-la-hipuxf3tesis-nula-h-0-ux3b20ux3b21ux3b220}{%
\paragraph{(f) El contraste de la hipótesis nula H 0 :
β0=β1=β2=0}\label{f-el-contraste-de-la-hipuxf3tesis-nula-h-0-ux3b20ux3b21ux3b220}}

\begin{Shaded}
\begin{Highlighting}[]
\NormalTok{g0 }\OtherTok{\textless{}{-}} \FunctionTok{lm}\NormalTok{(y }\SpecialCharTok{\textasciitilde{}} \DecValTok{0}\NormalTok{)}
\FunctionTok{anova}\NormalTok{(g0,lmod10)}
\end{Highlighting}
\end{Shaded}

\begin{verbatim}
## Analysis of Variance Table
## 
## Model 1: y ~ 0
## Model 2: y ~ x1 + x2
##   Res.Df     RSS Df Sum of Sq      F    Pr(>F)    
## 1      8 20336.0                                  
## 2      5    91.7  3     20244 368.15 2.773e-06 ***
## ---
## Signif. codes:  0 '***' 0.001 '**' 0.01 '*' 0.05 '.' 0.1 ' ' 1
\end{verbatim}

Se rechaza

\hypertarget{g-el-contraste-de-la-hipuxf3tesis-nula-h-0-ux3b2-1-ux3b2-2-0-utilizando-datos-originales.}{%
\paragraph{(g) El contraste de la hipótesis nula H 0 : β 1 = β 2 = 0
utilizando datos
originales.}\label{g-el-contraste-de-la-hipuxf3tesis-nula-h-0-ux3b2-1-ux3b2-2-0-utilizando-datos-originales.}}

\begin{Shaded}
\begin{Highlighting}[]
\NormalTok{g1}\OtherTok{\textless{}{-}}\FunctionTok{lm}\NormalTok{(y }\SpecialCharTok{\textasciitilde{}} \DecValTok{1}\NormalTok{)}
\FunctionTok{anova}\NormalTok{(g1,lmod10)}
\end{Highlighting}
\end{Shaded}

\begin{verbatim}
## Analysis of Variance Table
## 
## Model 1: y ~ 1
## Model 2: y ~ x1 + x2
##   Res.Df    RSS Df Sum of Sq      F    Pr(>F)    
## 1      7 3408.0                                  
## 2      5   91.7  2    3316.3 90.462 0.0001186 ***
## ---
## Signif. codes:  0 '***' 0.001 '**' 0.01 '*' 0.05 '.' 0.1 ' ' 1
\end{verbatim}

Se rechaza

\hypertarget{h-el-contraste-de-la-hipuxf3tesis-nula-h-0-ux3b2-1-ux3b2-2-0-utilizando-datos-en-desviaciones-respecto-a-la-media.}{%
\paragraph{(h) El contraste de la hipótesis nula H 0 : β 1 = β 2 = 0
utilizando datos en desviaciones respecto a la
media.}\label{h-el-contraste-de-la-hipuxf3tesis-nula-h-0-ux3b2-1-ux3b2-2-0-utilizando-datos-en-desviaciones-respecto-a-la-media.}}

\begin{Shaded}
\begin{Highlighting}[]
\NormalTok{g1s}\OtherTok{\textless{}{-}}\FunctionTok{lm}\NormalTok{(ys }\SpecialCharTok{\textasciitilde{}} \DecValTok{1}\NormalTok{)}
\FunctionTok{anova}\NormalTok{(g1s, lmod11)}
\end{Highlighting}
\end{Shaded}

\begin{verbatim}
## Analysis of Variance Table
## 
## Model 1: ys ~ 1
## Model 2: ys ~ 0 + x1s + x2s
##   Res.Df    RSS Df Sum of Sq      F   Pr(>F)    
## 1      7 3408.0                                 
## 2      6   91.7  1    3316.3 217.11 6.14e-06 ***
## ---
## Signif. codes:  0 '***' 0.001 '**' 0.01 '*' 0.05 '.' 0.1 ' ' 1
\end{verbatim}

\hypertarget{i-la-representaciuxf3n-gruxe1fica-de-una-regiuxf3n-de-confianza-del-95-para-ux3b2-1-y-ux3b2-2-.}{%
\paragraph{(i) La representación gráfica de una región de confianza del
95 \% para β 1 y β 2
.}\label{i-la-representaciuxf3n-gruxe1fica-de-una-regiuxf3n-de-confianza-del-95-para-ux3b2-1-y-ux3b2-2-.}}

\begin{Shaded}
\begin{Highlighting}[]
\FunctionTok{library}\NormalTok{(ellipse)}
\end{Highlighting}
\end{Shaded}

\begin{verbatim}
## 
## Attaching package: 'ellipse'
\end{verbatim}

\begin{verbatim}
## The following object is masked from 'package:graphics':
## 
##     pairs
\end{verbatim}

\begin{Shaded}
\begin{Highlighting}[]
\FunctionTok{plot}\NormalTok{(}\FunctionTok{ellipse}\NormalTok{(lmod10,}\DecValTok{2}\SpecialCharTok{:}\DecValTok{3}\NormalTok{),}\AttributeTok{type=}\StringTok{"l"}\NormalTok{)}
\FunctionTok{points}\NormalTok{(}\FunctionTok{coef}\NormalTok{(lmod10)[}\DecValTok{2}\NormalTok{], }\FunctionTok{coef}\NormalTok{(lmod10)[}\DecValTok{3}\NormalTok{], }\AttributeTok{pch=}\DecValTok{19}\NormalTok{)}
\FunctionTok{abline}\NormalTok{(}\AttributeTok{v=}\FunctionTok{confint}\NormalTok{(lmod10)[}\DecValTok{2}\NormalTok{,],}\AttributeTok{lty=}\DecValTok{2}\NormalTok{,}\AttributeTok{col=}\DecValTok{2}\NormalTok{)}
\FunctionTok{abline}\NormalTok{(}\AttributeTok{h=}\FunctionTok{confint}\NormalTok{(lmod10)[}\DecValTok{3}\NormalTok{,],}\AttributeTok{lty=}\DecValTok{2}\NormalTok{,}\AttributeTok{col=}\DecValTok{2}\NormalTok{)}
\end{Highlighting}
\end{Shaded}

\includegraphics{Regresion-simple-y-multiple_files/figure-latex/unnamed-chunk-59-1.pdf}

\hypertarget{j-el-contraste-individual-de-los-paruxe1metros-ux3b2-0-ux3b2-1-y-ux3b2-2-.}{%
\paragraph{(j) El contraste individual de los parámetros β 0 , β 1 y β 2
.}\label{j-el-contraste-individual-de-los-paruxe1metros-ux3b2-0-ux3b2-1-y-ux3b2-2-.}}

\begin{Shaded}
\begin{Highlighting}[]
\FunctionTok{summary}\NormalTok{(lmod10)}
\end{Highlighting}
\end{Shaded}

\begin{verbatim}
## 
## Call:
## lm(formula = y ~ x1 + x2)
## 
## Residuals:
##       1       2       3       4       5       6       7       8 
## -3.0583 -0.9777 -0.8233  3.3310  5.6690  2.8233 -5.0961 -1.8679 
## 
## Coefficients:
##             Estimate Std. Error t value Pr(>|t|)    
## (Intercept)   6.4700     3.3684   1.921    0.113    
## x1            6.5883     0.5015  13.137 4.56e-05 ***
## x2            0.2573     1.5458   0.166    0.874    
## ---
## Signif. codes:  0 '***' 0.001 '**' 0.01 '*' 0.05 '.' 0.1 ' ' 1
## 
## Residual standard error: 4.281 on 5 degrees of freedom
## Multiple R-squared:  0.9731, Adjusted R-squared:  0.9624 
## F-statistic: 90.46 on 2 and 5 DF,  p-value: 0.0001186
\end{verbatim}

Beta 1 es significativo

\hypertarget{k-el-contraste-de-la-hipuxf3tesis-nula-h0-ux3b21-10-ux3b22-.}{%
\paragraph{(k) El contraste de la hipótesis nula H0: β1 = 10 β2
.}\label{k-el-contraste-de-la-hipuxf3tesis-nula-h0-ux3b21-10-ux3b22-.}}

\begin{Shaded}
\begin{Highlighting}[]
\NormalTok{lmod12}\OtherTok{\textless{}{-}}\FunctionTok{lm}\NormalTok{(y }\SpecialCharTok{\textasciitilde{}} \FunctionTok{I}\NormalTok{(}\DecValTok{10}\SpecialCharTok{*}\NormalTok{x1 }\SpecialCharTok{+}\NormalTok{ x2))}
\FunctionTok{anova}\NormalTok{(lmod12,lmod10)}
\end{Highlighting}
\end{Shaded}

\begin{verbatim}
## Analysis of Variance Table
## 
## Model 1: y ~ I(10 * x1 + x2)
## Model 2: y ~ x1 + x2
##   Res.Df   RSS Df Sum of Sq      F Pr(>F)
## 1      6 92.87                           
## 2      5 91.65  1    1.2195 0.0665 0.8067
\end{verbatim}

Se acepta

\hypertarget{ejercicios-del-libro-de-faraway-1}{%
\subsection{Ejercicios del libro de
Faraway}\label{ejercicios-del-libro-de-faraway-1}}

\hypertarget{ejercicio-4-cap.-4-puxe1g.-57}{%
\paragraph{4. (∗) (Ejercicio 4 cap. 4 pág.
57)}\label{ejercicio-4-cap.-4-puxe1g.-57}}

The dataset mdeaths reports the number of deaths from lung diseases for
men in the UK from 1974 to 1979.

\hypertarget{a-make-an-appropriate-plot-of-the-data.-at-what-time-of-year-are-deaths-most-likely-to-occur}{%
\paragraph{(a) Make an appropriate plot of the data. At what time of
year are deaths most likely to
occur?}\label{a-make-an-appropriate-plot-of-the-data.-at-what-time-of-year-are-deaths-most-likely-to-occur}}

\begin{Shaded}
\begin{Highlighting}[]
\FunctionTok{library}\NormalTok{(datasets)}
\FunctionTok{data}\NormalTok{(UKLungDeaths)}
\FunctionTok{plot}\NormalTok{(mdeaths)}
\end{Highlighting}
\end{Shaded}

\includegraphics{Regresion-simple-y-multiple_files/figure-latex/unnamed-chunk-62-1.pdf}
Hay más muertes a primeros de año

\hypertarget{b-fit-an-autoregressive-model-of-the-same-form-used-for-the-airline-data.-are-all-the-predictors-statistically-significant}{%
\paragraph{(b) Fit an autoregressive model of the same form used for the
airline data. Are all the predictors statistically
significant?}\label{b-fit-an-autoregressive-model-of-the-same-form-used-for-the-airline-data.-are-all-the-predictors-statistically-significant}}

\begin{Shaded}
\begin{Highlighting}[]
\NormalTok{lagdf }\OtherTok{\textless{}{-}} \FunctionTok{embed}\NormalTok{(}\FunctionTok{as.vector}\NormalTok{(mdeaths),}\DecValTok{14}\NormalTok{)}
\FunctionTok{colnames}\NormalTok{(lagdf) }\OtherTok{\textless{}{-}} \FunctionTok{c}\NormalTok{(}\StringTok{"y"}\NormalTok{,}\FunctionTok{paste0}\NormalTok{(}\StringTok{"lag"}\NormalTok{,}\DecValTok{1}\SpecialCharTok{:}\DecValTok{13}\NormalTok{))}
\NormalTok{lagdf }\OtherTok{\textless{}{-}} \FunctionTok{data.frame}\NormalTok{(lagdf)}
\NormalTok{armod }\OtherTok{\textless{}{-}} \FunctionTok{lm}\NormalTok{(y }\SpecialCharTok{\textasciitilde{}}\NormalTok{ lag1 }\SpecialCharTok{+}\NormalTok{ lag12 }\SpecialCharTok{+}\NormalTok{ lag13, }\AttributeTok{data=}\NormalTok{lagdf)}
\FunctionTok{summary}\NormalTok{(armod)}
\end{Highlighting}
\end{Shaded}

\begin{verbatim}
## 
## Call:
## lm(formula = y ~ lag1 + lag12 + lag13, data = lagdf)
## 
## Residuals:
##     Min      1Q  Median      3Q     Max 
## -762.71  -81.13  -21.12   61.76  724.06 
## 
## Coefficients:
##             Estimate Std. Error t value Pr(>|t|)    
## (Intercept)  58.1985   120.7358   0.482   0.6317    
## lag1          0.2501     0.1327   1.885   0.0647 .  
## lag12         0.5356     0.1179   4.542 3.09e-05 ***
## lag13         0.1512     0.1386   1.091   0.2801    
## ---
## Signif. codes:  0 '***' 0.001 '**' 0.01 '*' 0.05 '.' 0.1 ' ' 1
## 
## Residual standard error: 238.7 on 55 degrees of freedom
## Multiple R-squared:   0.73,  Adjusted R-squared:  0.7153 
## F-statistic: 49.56 on 3 and 55 DF,  p-value: 1.19e-15
\end{verbatim}

\hypertarget{c-use-the-model-to-predict-the-number-of-deaths-in-january-1980-along-with-a-95-prediction-interval.}{%
\paragraph{c) Use the model to predict the number of deaths in January
1980 along with a 95\% prediction
interval.}\label{c-use-the-model-to-predict-the-number-of-deaths-in-january-1980-along-with-a-95-prediction-interval.}}

\begin{Shaded}
\begin{Highlighting}[]
\NormalTok{lagdf[}\FunctionTok{nrow}\NormalTok{(lagdf),]}
\end{Highlighting}
\end{Shaded}

\begin{verbatim}
##       y lag1 lag2 lag3 lag4 lag5 lag6 lag7 lag8 lag9 lag10 lag11 lag12 lag13
## 59 1341 1294 1081  940  975 1056 1075 1215 1531 1846  1820  2263  1812  1110
\end{verbatim}

\begin{Shaded}
\begin{Highlighting}[]
\FunctionTok{predict}\NormalTok{(armod, }\FunctionTok{data.frame}\NormalTok{(}\AttributeTok{lag1=}\DecValTok{1341}\NormalTok{, }\AttributeTok{lag12=}\DecValTok{2263}\NormalTok{, }\AttributeTok{lag13=}\DecValTok{1812}\NormalTok{),}\AttributeTok{interval=}\StringTok{"prediction"}\NormalTok{)}
\end{Highlighting}
\end{Shaded}

\begin{verbatim}
##        fit      lwr      upr
## 1 1879.599 1359.725 2399.474
\end{verbatim}

\hypertarget{d-use-your-answer-from-the-previous-question-to-compute-a-prediction-and-interval-for-february-1980.}{%
\paragraph{(d) Use your answer from the previous question to compute a
prediction and interval for February
1980.}\label{d-use-your-answer-from-the-previous-question-to-compute-a-prediction-and-interval-for-february-1980.}}

\begin{Shaded}
\begin{Highlighting}[]
\FunctionTok{predict}\NormalTok{(armod, }\FunctionTok{data.frame}\NormalTok{(}\AttributeTok{lag1=}\FloatTok{1879.599}\NormalTok{, }\AttributeTok{lag12=}\DecValTok{1820}\NormalTok{, }\AttributeTok{lag13=}\DecValTok{2263}\NormalTok{),}\AttributeTok{interval=}\StringTok{"prediction"}\NormalTok{)}
\end{Highlighting}
\end{Shaded}

\begin{verbatim}
##        fit     lwr      upr
## 1 1845.247 1345.87 2344.625
\end{verbatim}

\hypertarget{e-compute-the-fitted-values.-plot-these-against-the-observed-values.-note-that-you-will-need-to-select-the-appropriate-observed-values.-do-you-think-the-accuracy-of-predictions-will-be-the-same-for-all-months-of-the-year}{%
\paragraph{(e) Compute the fitted values. Plot these against the
observed values. Note that you will need to select the appropriate
observed values. Do you think the accuracy of predictions will be the
same for all months of the
year?}\label{e-compute-the-fitted-values.-plot-these-against-the-observed-values.-note-that-you-will-need-to-select-the-appropriate-observed-values.-do-you-think-the-accuracy-of-predictions-will-be-the-same-for-all-months-of-the-year}}

\begin{Shaded}
\begin{Highlighting}[]
\FunctionTok{plot}\NormalTok{(lagdf}\SpecialCharTok{$}\NormalTok{y, }\AttributeTok{type=}\StringTok{"l"}\NormalTok{, }\AttributeTok{xlim=}\FunctionTok{c}\NormalTok{(}\DecValTok{0}\NormalTok{,}\DecValTok{62}\NormalTok{), }\AttributeTok{ylab=}\StringTok{"deaths"}\NormalTok{)}
\FunctionTok{lines}\NormalTok{(}\FunctionTok{predict}\NormalTok{(armod), }\AttributeTok{lty=}\DecValTok{2}\NormalTok{)}
\end{Highlighting}
\end{Shaded}

\includegraphics{Regresion-simple-y-multiple_files/figure-latex/unnamed-chunk-66-1.pdf}

\begin{Shaded}
\begin{Highlighting}[]
\NormalTok{pred.int }\OtherTok{\textless{}{-}} \FunctionTok{predict}\NormalTok{(armod, }\AttributeTok{interval =} \StringTok{"prediction"}\NormalTok{)}
\end{Highlighting}
\end{Shaded}

\begin{verbatim}
## Warning in predict.lm(armod, interval = "prediction"): predictions on current data refer to _future_ responses
\end{verbatim}

\begin{Shaded}
\begin{Highlighting}[]
\FunctionTok{plot}\NormalTok{(pred.int[,}\DecValTok{3}\NormalTok{]}\SpecialCharTok{{-}}\NormalTok{pred.int[,}\DecValTok{2}\NormalTok{], }\AttributeTok{type=}\StringTok{"l"}\NormalTok{, }\AttributeTok{ylab=}\StringTok{"Anchura del intervalo"}\NormalTok{)}
\end{Highlighting}
\end{Shaded}

\includegraphics{Regresion-simple-y-multiple_files/figure-latex/unnamed-chunk-67-1.pdf}

\begin{Shaded}
\begin{Highlighting}[]
\FunctionTok{which.max}\NormalTok{(pred.int[,}\DecValTok{3}\NormalTok{]}\SpecialCharTok{{-}}\NormalTok{pred.int[,}\DecValTok{2}\NormalTok{])}
\end{Highlighting}
\end{Shaded}

\begin{verbatim}
## 26 
## 26
\end{verbatim}

\hypertarget{ejercicio-5-cap.-4-puxe1g.-58}{%
\paragraph{5. (∗) (Ejercicio 5 cap. 4 pág.
58)}\label{ejercicio-5-cap.-4-puxe1g.-58}}

For the fat data used in this chapter, a smaller model using only age,
weight, height and abdom was proposed on the grounds that these
predictors are either known by the individual or easily measured.

\hypertarget{a-compare-this-model-to-the-full-thirteen-predictor-model-used-earlier-in-the-chapter.-is-it-justifiable-to-use-the-smaller-model}{%
\paragraph{(a) Compare this model to the full thirteen-predictor model
used earlier in the chapter. Is it justifiable to use the smaller
model?}\label{a-compare-this-model-to-the-full-thirteen-predictor-model-used-earlier-in-the-chapter.-is-it-justifiable-to-use-the-smaller-model}}

\begin{Shaded}
\begin{Highlighting}[]
\FunctionTok{data}\NormalTok{(fat,}\AttributeTok{package=}\StringTok{"faraway"}\NormalTok{)}

\NormalTok{lmod13}\OtherTok{\textless{}{-}}\FunctionTok{lm}\NormalTok{(brozek }\SpecialCharTok{\textasciitilde{}}\NormalTok{ age }\SpecialCharTok{+}\NormalTok{ weight }\SpecialCharTok{+}\NormalTok{ height }\SpecialCharTok{+}\NormalTok{ neck }\SpecialCharTok{+}\NormalTok{ chest }\SpecialCharTok{+}\NormalTok{ abdom }\SpecialCharTok{+}
\NormalTok{               hip }\SpecialCharTok{+}\NormalTok{ thigh }\SpecialCharTok{+}\NormalTok{ knee }\SpecialCharTok{+}\NormalTok{ ankle }\SpecialCharTok{+}\NormalTok{ biceps }\SpecialCharTok{+}\NormalTok{ forearm }\SpecialCharTok{+}\NormalTok{ wrist, }\AttributeTok{data=}\NormalTok{fat)}

\NormalTok{lmod14}\OtherTok{\textless{}{-}}\FunctionTok{lm}\NormalTok{(brozek }\SpecialCharTok{\textasciitilde{}}\NormalTok{ age }\SpecialCharTok{+}\NormalTok{  weight }\SpecialCharTok{+}\NormalTok{  height }\SpecialCharTok{+}\NormalTok{ abdom, }\AttributeTok{data=}\NormalTok{fat)}

\FunctionTok{anova}\NormalTok{(lmod13, lmod14)}
\end{Highlighting}
\end{Shaded}

\begin{verbatim}
## Analysis of Variance Table
## 
## Model 1: brozek ~ age + weight + height + neck + chest + abdom + hip + 
##     thigh + knee + ankle + biceps + forearm + wrist
## Model 2: brozek ~ age + weight + height + abdom
##   Res.Df    RSS Df Sum of Sq      F   Pr(>F)   
## 1    238 3785.1                                
## 2    247 4205.0 -9    -419.9 2.9336 0.002558 **
## ---
## Signif. codes:  0 '***' 0.001 '**' 0.01 '*' 0.05 '.' 0.1 ' ' 1
\end{verbatim}

Es significativo, así que no podemos usar el modelo simple.

\hypertarget{b-compute-a-95-prediction-interval-for-median-predictor-values-and-compare-to-the-results-to-the-interval-for-the-full-model.-do-the-intervals-differ-by-a-practically-important-amount}{%
\paragraph{(b) Compute a 95\% prediction interval for median predictor
values and compare to the results to the interval for the full model. Do
the intervals differ by a practically important
amount?}\label{b-compute-a-95-prediction-interval-for-median-predictor-values-and-compare-to-the-results-to-the-interval-for-the-full-model.-do-the-intervals-differ-by-a-practically-important-amount}}

\begin{Shaded}
\begin{Highlighting}[]
\NormalTok{medianas }\OtherTok{\textless{}{-}} \FunctionTok{apply}\NormalTok{(fat[,}\DecValTok{4}\SpecialCharTok{:}\DecValTok{18}\NormalTok{],}\DecValTok{2}\NormalTok{,median)}
\FunctionTok{predict}\NormalTok{(lmod14, }\AttributeTok{newdata =} \FunctionTok{data.frame}\NormalTok{(}\AttributeTok{age=}\NormalTok{medianas[}\DecValTok{1}\NormalTok{],}\AttributeTok{weight=}\NormalTok{medianas[}\DecValTok{2}\NormalTok{],}\AttributeTok{height=}\NormalTok{medianas[}\DecValTok{3}\NormalTok{],}\AttributeTok{abdom=}\NormalTok{medianas[}\DecValTok{8}\NormalTok{]), }\AttributeTok{interval=}\StringTok{"prediction"}\NormalTok{)}
\end{Highlighting}
\end{Shaded}

\begin{verbatim}
##          fit      lwr      upr
## age 17.84028 9.696631 25.98392
\end{verbatim}

\begin{Shaded}
\begin{Highlighting}[]
\FunctionTok{predict}\NormalTok{(lmod13, }\AttributeTok{newdata =} \FunctionTok{as.data.frame}\NormalTok{(}\FunctionTok{t}\NormalTok{(medianas[}\FunctionTok{c}\NormalTok{(}\DecValTok{1}\SpecialCharTok{:}\DecValTok{3}\NormalTok{,}\DecValTok{6}\SpecialCharTok{:}\DecValTok{15}\NormalTok{)])), }\AttributeTok{interval=}\StringTok{"prediction"}\NormalTok{)}
\end{Highlighting}
\end{Shaded}

\begin{verbatim}
##        fit     lwr      upr
## 1 17.49322 9.61783 25.36861
\end{verbatim}

\hypertarget{c-for-the-smaller-model-examine-all-the-observations-from-case-numbers-25-to-50.-which-two-observations-seem-particularly-anomalous}{%
\paragraph{(c) For the smaller model, examine all the observations from
case numbers 25 to 50. Which two observations seem particularly
anomalous?}\label{c-for-the-smaller-model-examine-all-the-observations-from-case-numbers-25-to-50.-which-two-observations-seem-particularly-anomalous}}

\begin{Shaded}
\begin{Highlighting}[]
\FunctionTok{plot}\NormalTok{(lmod14, }\AttributeTok{which=}\DecValTok{5}\NormalTok{)}
\end{Highlighting}
\end{Shaded}

\includegraphics{Regresion-simple-y-multiple_files/figure-latex/unnamed-chunk-70-1.pdf}

\hypertarget{d-recompute-the-95-prediction-interval-for-median-predictor-values-after-these-two-anomalous-cases-have-been-excluded-from-the-data.-did-this-make-much-difference-to-the-outcome}{%
\paragraph{(d) Recompute the 95\% prediction interval for median
predictor values after these two anomalous cases have been excluded from
the data. Did this make much difference to the
outcome?}\label{d-recompute-the-95-prediction-interval-for-median-predictor-values-after-these-two-anomalous-cases-have-been-excluded-from-the-data.-did-this-make-much-difference-to-the-outcome}}

\begin{Shaded}
\begin{Highlighting}[]
\NormalTok{lmod15 }\OtherTok{\textless{}{-}} \FunctionTok{lm}\NormalTok{(brozek }\SpecialCharTok{\textasciitilde{}}\NormalTok{ age }\SpecialCharTok{+}\NormalTok{ weight }\SpecialCharTok{+}\NormalTok{ height }\SpecialCharTok{+}\NormalTok{ abdom, }\AttributeTok{data=}\NormalTok{fat[}\SpecialCharTok{{-}}\FunctionTok{c}\NormalTok{(}\DecValTok{39}\NormalTok{,}\DecValTok{42}\NormalTok{),])}
\NormalTok{medianas }\OtherTok{\textless{}{-}} \FunctionTok{apply}\NormalTok{(fat[}\SpecialCharTok{{-}}\FunctionTok{c}\NormalTok{(}\DecValTok{39}\NormalTok{,}\DecValTok{42}\NormalTok{),}\DecValTok{4}\SpecialCharTok{:}\DecValTok{18}\NormalTok{],}\DecValTok{2}\NormalTok{,median)}

\FunctionTok{predict}\NormalTok{(lmod15, }\AttributeTok{newdata =} \FunctionTok{data.frame}\NormalTok{(}\AttributeTok{age=}\NormalTok{medianas[}\DecValTok{1}\NormalTok{],}\AttributeTok{weight=}\NormalTok{medianas[}\DecValTok{2}\NormalTok{],}\AttributeTok{height=}\NormalTok{medianas[}\DecValTok{3}\NormalTok{],}\AttributeTok{abdom=}\NormalTok{medianas[}\DecValTok{8}\NormalTok{]), }\AttributeTok{interval=}\StringTok{"prediction"}\NormalTok{)}
\end{Highlighting}
\end{Shaded}

\begin{verbatim}
##         fit      lwr      upr
## age 17.9033 9.887851 25.91874
\end{verbatim}

\end{document}
